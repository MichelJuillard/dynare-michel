\chapter{Introduction} \label{ch:intro}

Welcome to Dynare! \\

\section{About This Guide: Approach and Structure}
The Dynare User Guide \textbf{aims to help you master Dynare}'s main functionalities, from getting started to implementing advanced features. To do so, this guide is structured around concrete examples and offers practical advice. To root this understanding more deeply, it also gives some background on Dynare's algorithms, methodologies and underlying theory. Thus, its secondary function is to \textbf{serve as a basic primer} on DSGE model solving and Bayesian estimation. \\

This guide will focus on the most common and useful features of the program, thus emphasizing \textbf{depth over breadth}. The idea is to get you to use 90\% of the program well and then tell you where else to look if you're interested in learning more.\\

This guide is primarily intended for the \textbf{advanced economist} needing a powerful and flexible program to facilitate his or her research activities in a variety of fields. Consequently, the sophisticated computer programmer and the computational economics specialist may not find this guide sufficiently detailed. \\

We recognize that the ``advanced economist'' may be either a beginning or intermediate user of Dynare. This guide is written to accommodate both. If you're \textbf{new to Dynare}, we recommend starting with chapters \ref{ch:solbase} and \ref{ch:estbase}, which introduce the program's basic features to solve and estimate DSGE models, respectively. To do so, these chapters lead you through a complete hands-on example, which we recommend following from A to Z in order to ``\textbf{learn by doing}''. Once you have read these two chapters, you will know the crux of Dynare's functionality and (hopefully!) feel comfortable using Dynare for your own work. At that point, though, you will probably find yourself coming back to the User Guide to skim over some of the content in the advanced chapters to iron out details and potential complications you may run into.\\

If, instead, you're an \textbf{intermediate user} of Dynare, you will most likely find the advanced chapters (\ref{ch:soladv} and \ref{ch:estadv}) more appropriate. These chapters cover more advanced features of Dynare and more complicated usage scenarios. The presumption is that you would skip around these chapters, focusing on the topics most applicable to your needs and curiosity. Examples are therefore more concise and specific to each feature; these chapters read a bit more like a reference manual.\\

We also recognize that you probably have had repeated if not active exposure to programming and are likely to have a strong economic background. Thus, a black box solution to your needs is inadequate. To hopefully address this issue, the User Guide goes into some depth in covering the \textbf{theoretical underpinnings and methodologies that Dynare follows} to solve and estimate DSGE models. These are available in the ``Behind the Scenes of Dynare'' chapters (\ref{ch:solbeh} and \ref{ch:estbeh}). These chapters can also serve as a \textbf{basic primer} if you are new to the practice of DSGE model solving and Bayesian estimation. \\

Finally, besides breaking up content into short chapters, we've introduced two different \textbf{markers} throughout the Guide to help streamline your reading.
\begin{itemize}
\item \textbf{\textsf{TIP!}} introduces advice to help you work more efficiently with Dynare or solve common problems.  
\item \textbf{\textsf{NOTE!}} is used to draw your attention to particularly important information you should keep in mind when using Dynare. 
\end{itemize} 


\section{What is Dynare?}
Before we dive into the thick of the ``trees'', let's have a look at the ``forest'' from the top \ldots just what is Dynare? \\

\textbf{Dynare is a powerful and highly customizable engine, with an intuitive front-end interface, to solve, simulate and estimate DSGE models}. \\

In slightly less flowery words, it is a pre-processor and a collection of Matlab routines that has the great advantage of reading DSGE model equations written almost as if they were in an academic paper. This not only facilitates the preparation of a model for processing by Dynare, but also enables you to easily share your code with others as it will be comprehensible by all.\\
\begin{figure}
\begin{center}
\includegraphics[width=1.0\textwidth]{P_DynareStruct2}
\end{center}
\caption[Dynare, a bird's eyeview]{The .mod file being read by the Dynare pre-processor, which then calls the relevant Matlab routines to carry out the desired operations and display the results.}
\label{fig:dyn}
\end{figure}

Figure \ref{fig:dyn} gives you an overview of the way Dynare works. Basically, the model, as well as its related attributes (e.g. shock structure), is written equation by equation in your preferred text editor. The resulting file (called the .mod file) is then called from Matlab. This initiates the Dynare pre-processor, which translates the .mod file into suitable input (i.e. intermediary Matlab or C files) for the Matlab routines used to either solve or estimate the model.  Finally, results are presented in Matlab. Some more details on the internal files generated by Dynare are given in section \ref{sec:dynfiles}. \\

Each of these steps will become clear as you read through the User Guide, but for now it may be helpful to summarize \textbf{what Dynare is able to do}:
\begin{itemize}
\item compute the steady state of a model
\item compute the solution of deterministic models
\item compute the first and second order approximation to solutions of stochastic models
\item estimate parameters of DSGE models using either a maximum likelihood or a Bayesian approach
\item compute optimal policies in linear-quadratic models
\end{itemize}


\section{Additional Sources of Help}
While this User Guide tries to be as complete and thorough as possible, at times you will almost certainly want to browse other material for help. At your disposal, you have the following additional resources:
\begin{itemize}
\item \href{http://www.cepremap.cnrs.fr/juillard/mambo/download/manual/index.html}{\textbf{Reference Manual}}: this manual covers all Dynare commands, giving a clear definition and explanation of usage for each. The User Guide will often introduce you to a command in a rather loose manner (mainly through examples). Thus, reading corresponding command descriptions in the Reference Manual is a good idea to cover all relevant details. 
\item \href{http://www.cepremap.cnrs.fr/juillard/mambo/index.php?option=com_content&task=category&sectionid=11&id=96&Itemid=89}{\textbf{Official online examples}}: the Dynare website includes other (often well-documented) examples of .mod files covering models and methodologies introduced in recent papers. 
\item \href{http://www.cepremap.cnrs.fr/juillard/mambo/index.php?option=com_forum&Itemid=95&page=viewforum&f=2&sid=10290a11eb7a48243971159f5b86f83e}{\textbf{Open online examples}}: this page lists .mod files posted by users covering a wide variety of examples. 
\item \href{http://www.dynare.org/phpBB3}{\textbf{Dynare forums}}: this lively online discussion forum allows you to ask your questions openly and read threads from others who might have run into similar difficulties. 
\item \href{http://www.cepremap.cnrs.fr/juillard/mambo/index.php?option=com_content&task=section&id=3&Itemid=40}{\textbf{Frequently Asked Questions}} (FAQ): this section of the Dynare site emphasizes a few of the most popular questions in the forums. 
\item \href{http://www.dsge.net}{\textbf{DSGE.net}}: this website, run by members of the Dynare team, is a resource for all scholars working in the field of DSGE modeling. Besides allowing you to stay up to date with the most recent papers and possibly make new contacts, it conveniently lists conferences, workshops and seminars that may be of interest. 
\end{itemize}

\section{Nomenclature}
To end this introduction and avoid confusion in what follows, it is worthwhile to \textbf{define a few terms}. Many of these are shared with the Reference Manual. 
\begin{itemize}
\item \textbf{Integer} indicates an integer number.
\item \textbf{Double} indicates a double precision number. The following syntaxes are valid: 1.1e3, 1.1E3, 1.1E-3, 1.1d3, 1.1D3
\item \textbf{Expression} indicates a mathematical expression valid in the underlying language (e.g. Matlab).
\item \textbf{Variable name} indicates a variable name. \textbf{\textsf{NOTE!}} Variable names can be comprised of any combination of alphanumeric characters and underscores, as long as the first character is a letter. All other characters, including accents and \textbf{spaces}, are forbidden. 
\item \textbf{Parameter name} indicates a parameter name which must follow the same naming conventions as above. 
\item \textbf{Filename} indicates a valid file name on your operating system. Note that Matlab requires that names of files or functions begin with a letter; this concerns your Dynare .mod files.
\item \textbf{Command} is an instruction to Dynare or other program when specified. 
\item \textbf{Options} or optional arguments for a command are listed in square brackets \mbox{[ ]} unless otherwise noted. If, for instance, the option must be specified in parenthesis in Dynare, it will show up in the Guide as [(option)].
\item \textbf{\texttt{Typewritten text}} indicates text as it should appear in Dynare code.
\end{itemize}

\section{Dynare Version 4: What's New and Backward Compatibility}
This guide has been written for Dynare version 4. With respect to version 3, this new version introduces several important features, improvements, function optimizations and bug fixes. The major new features are the following: 
\begin{itemize}
\item Analytical derivatives are now used everywhere (for instance, in the Newton algorithm for deterministic models and in the linearizations necessary to solve stochastic models). This increases computational speed significantly. The drawback is that Dynare can now handle only a limited set of functions although, in nearly all economic applications, this should not be a constraint. 
\item Variables and parameters are now kept in the order in which they are declared whenever displayed (and when used internally) by Dynare. Recall that in version 3, variables and parameters were variably displayed in either their order of declaration or in alphabetical order. \textbf{\textsf{NOTE!}} This may cause programs that run off of version 3 output to function incorrectly after switching to Dynare version 4.
\item The names of many internal variables and the organization of output variables has changed. These are enumerated in detail in the relevant chapters. The names of the files internally generated by Dynare have also changed. (** more on this when explaining internal file structure - TBD)
\item The syntax for the external steady state file has changed. See section \ref{sec:findsteady} for more details. \textbf{\textsf{NOTE!}} You will unfortunately have to slightly amend any old steady state files you may have written. 
\item Speed. Several large-scale improvements have been implemented to speed up Dynare. This should be most noticeable when solving deterministic models, but also apparent in other functionality.
\end{itemize}