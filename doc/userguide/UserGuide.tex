\documentclass[a4paper,11pt]{memoir}
\maxsecnumdepth{subsection} 
\setsecnumdepth{subsection} 
\maxtocdepth{subsection} 
\settocdepth{subsection}
\usepackage{amsmath}
\usepackage{amsfonts}
\usepackage{amssymb}
\usepackage{rotating}
\usepackage{graphicx}
\usepackage[applemac]{inputenc}
\usepackage{amsbsy,enumerate}
\usepackage{natbib}
\usepackage{pdfpages}
%\usepackage{makeidx}
%\usepackage[pdftex]{hyperref}
%\usepackage{hyperref}
%\hypersetup{colorlinks=false}
\usepackage{hyperref} 
\hypersetup{colorlinks,% 
citecolor=green,% 
filecolor=magenta,% 
linkcolor=red,% 
urlcolor=cyan,% 
pdftex} 


\newtheorem{definition}{Definition} [chapter]
\newtheorem{lemma}{Lemma} [chapter]
\newtheorem{property}{Property} [chapter]
\newtheorem{theorem}{Theorem} [chapter]
\newtheorem{remark}{Remark} [chapter]

%\makeindex

\begin{document}
\frontmatter 

\includepdf{Graphics/DynareTitle.pdf}

\title{Dynare v4 - User Guide \\�Public beta version}
\author{\\ \\ \\ \\ \\ \\ \\ \\ \\ \\ \\  Tommaso Mancini Griffoli\\ tommaso.mancini@stanfordalumni.org}
\date{This draft: January 2013}

\maketitle

\thispagestyle{empty}

\newpage

~\vfill

Copyright � 2007-2013 Tommaso Mancini Griffoli

\bigskip

Permission is granted to copy, distribute and/or modify this document under the terms of the GNU Free Documentation License, Version 1.3 or any later version published by the Free Software Foundation; with no Invariant Sections, no Front-Cover Texts, and no Back-Cover Texts.

\bigskip

A copy of the license can be found at: \url{http://www.gnu.org/licenses/fdl.txt}

\vfill

\newpage

\tableofcontents
\listoffigures

\chapter*{Work in Progress!}
This is the second version of the Dynare User Guide which is still work in progress! This means two things. First, please read this with a critical eye and send me comments! Are some areas unclear? Is anything plain wrong? Are some sections too wordy, are there enough examples, are these clear? On the contrary, are there certain parts that just click particularly well? How can others be improved? I'm very interested to get your feedback. \\

The second thing that a work in progress manuscript comes with is a few internal notes. These are mostly placeholders for future work, notes to myself or others of the Dynare development team, or at times notes to you - our readers - to highlight a feature not yet fully stable. Any such notes are marked with two stars (**). \\

Thanks very much for your patience and good ideas. Please write either direclty to myself: tommaso.mancini@stanfordalumni.org, or \textbf{preferably on the Dynare Documentation Forum} available in the \href{http://www.dynare.org/phpBB3}{Dynare Forums}. 

\chapter*{Contacts and Credits} \label{ch:contacts}
Dynare was originally developed by Michel Juillard in Paris, France. Currently, the development team of Dynare is composed of

\begin{itemize}
\item St�phane Adjemian \texttt{<stephane.adjemian@univ-lemans.fr>}
\item Houtan Bastani \texttt{<houtan@dynare.org>}
\item Michel Juillard \texttt{<michel.juillard@mjui.fr>}
\item Fr�d�ric Karam� \texttt{<frederic.karame@univ-lemans.fr>}
\item Junior Maih \texttt{<junior.maih@gmail.com>}
\item Ferhat Mihoubi \texttt{<ferhat.mihoubi@cepremap.org>}
\item George Perendia \texttt{<george@perendia.orangehome.co.uk>}
\item Johannes Pfeifer \texttt{<jpfeifer@gmx.de>}
\item Marco Ratto \texttt{<marco.ratto@jrc.ec.europa.eu>}
\item S�bastien Villemot \texttt{<sebastien@dynare.org>}
\end{itemize}

Several parts of Dynare use or have strongly benefited from publicly available programs by G. Anderson, F. Collard, L. Ingber, O. Kamenik, P. Klein, S. Sakata, F. Schorfheide, C. Sims, P. Soederlind and R. Wouters.

Finally, the development of Dynare could not have come such a long ways withough an active community of users who continually pose questions, report bugs and suggest new features. The help of this community is gratefully acknowledged.\\

The email addresses above are provided in case you wish to contact any one of the authors of Dynare directly. We nonetheless encourage you to first use the \href{http://www.dynare.org/phpBB3}{Dynare forums} to ask your questions so that other users can benefit from them as well; remember, almost no question is specific enough to interest just one person, and yours is not the exception! 

\mainmatter

\include{ch-intro}
\include{ch-inst}
\include{ch-solbase}
\include{ch-soladv}
\include{ch-estbase}
\include{ch-estadv}
\include{ch-solbeh}
\include{ch-estbeh}
\include{ch-ramsey}
\include{ch-trouble}

\backmatter

\bibliography{DynareBib}
\bibliographystyle{elsarticle-harv}
%\printindex

\end{document}

