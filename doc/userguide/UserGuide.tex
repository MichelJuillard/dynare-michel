\documentclass[a4paper,11pt,titlepage]{memoir}
\maxsecnumdepth{subsection} 
\setsecnumdepth{subsection} 
\maxtocdepth{subsection} 
\settocdepth{subsection}
\usepackage{amsmath}
\usepackage{amsfonts}
\usepackage{amssymb}
\usepackage{rotating}
\usepackage{graphicx}
\usepackage[applemac]{inputenc}
\usepackage{amsbsy,enumerate}
\usepackage{natbib}
%\usepackage{makeidx}
%\usepackage[pdftex]{hyperref}
%\usepackage{hyperref}
%\hypersetup{colorlinks=false}
\usepackage{hyperref} 
\hypersetup{colorlinks,% 
citecolor=green,% 
filecolor=magenta,% 
linkcolor=red,% 
urlcolor=cyan,% 
pdftex} 


\newtheorem{definition}{Definition} [chapter]
\newtheorem{lemma}{Lemma} [chapter]
\newtheorem{property}{Property} [chapter]
\newtheorem{theorem}{Theorem} [chapter]
\newtheorem{remark}{Remark} [chapter]

%\makeindex

\begin{document}
\frontmatter 

\title{Dynare v4 - User Guide \\�Public beta version}
\author{\\ \\ \\ \\ \\ \\ \\ \\ \\ \\ \\  Tommaso Mancini Griffoli\\ tommaso.mancini@stanfordalumni.org}
\date{This draft: March 2007}

\maketitle

%\newpage

\tableofcontents
\listoffigures

\chapter*{Work in Progress!}
This is the second version of the Dynare User Guide which is still work in progress! This means two things. First, please read this with a critical eye and send me comments! Are some areas unclear? Is anything plain wrong? Are some sections too wordy, are there enough examples, are these clear? On the contrary, are there certain parts that just click particularly well? How can others be improved? I'm very interested to get your feedback. \\

The second thing that a work in progress manuscript comes with is a few internal notes. These are mostly placeholders for future work, notes to myself or others of the Dynare development team, or at times notes to you - our readers - to highlight a feature not yet fully stable. Any such notes are marked with two stars (**). \\

Thanks very much for your patience and good ideas. Please write either direclty to myself: tommaso.mancini@stanfordalumni.org, or \textbf{preferably on the Dynare Documentation Forum} available in the Forum section of the \href{http://www.cepremap.cnrs.fr/juillard/mambo/index.php?option=com_forum&Itemid=95}{Dynare website}. 

\chapter*{Contacts and Credits} \label{ch:contacts}
Dynare was originally developed by Michel Juillard in Paris, France. Currently, the development team of Dynare is composed of

\begin{itemize}
\item St�phane Adjemian (stephane.adjemian``AT''ens.fr)
\item Michel Juillard (michel.juillard``AT''ens.fr)
\item Ferhat Mihoubi (ferhat.mihoubi``AT''univ-evry.fr)
\item Ondra Kamenik (ondra.kamenik"AT"volny.cz)
\item Marco Ratto (marco.ratto"AT"jrc.it)
\item S�bastien Villemot (sebastien.villemot``AT''ens.fr)
\end{itemize}

Several parts of Dynare use or have strongly benefitted from publicly available programs by F. Collard, L. Ingber, P. Klein, S. Sakata, F. Schorfheide, C. Sims, P. Soederlind and R. Wouters. \\

Finally, the development of Dynare could not have come such a long ways withough an active community of users who continually pose questions, report bugs and suggest new features. The help of this community is gratefully acknowledged.\\

The email addresses above are provided in case you wish to contact any one of the authors of Dynare directly. We nonetheless encourage you to first use the \href{http://www.cepremap.cnrs.fr/juillard/mambo/index.php?option=com_forum&Itemid=95}{Dynare forums} to ask your questions so that other users can benefit from them as well; remember, almost no question is specific enough to interest just one person, and yours is not the exception! 

\mainmatter

\chapter{Introduction} \label{ch:intro}

Welcome to Dynare! \\

\section{About This Guide: Approach and Structure}
The Dynare User Guide \textbf{aims to help you master Dynare}'s main functionalities, from getting started to implementing advanced features. To do so, this guide is structured around concrete examples and offers practical advice. To root this understanding more deeply, it also gives some background on Dynare's algorithms, methodologies and underlying theory. Thus, its secondary function is to \textbf{serve as a basic primer} on DSGE model solving and Bayesian estimation. \\

This guide will focus on the most common and useful features of the program, thus emphasizing \textbf{depth over breadth}. The idea is to get you to use 90\% of the program well and then tell you where else to look if you're interested in learning more.\\

This guide is primarily intended for the \textbf{advanced economist} needing a powerful and flexible program to facilitate his or her research activities in a variety of fields. Consequently, the sophisticated computer programmer and the computational economics specialist may not find this guide sufficiently detailed. \\

We recognize that the ``advanced economist'' may be either a beginning or intermediate user of Dynare. This guide is written to accommodate both. If you're \textbf{new to Dynare}, we recommend starting with chapters \ref{ch:solbase} and \ref{ch:estbase}, which introduce the program's basic features to solve and estimate DSGE models, respectively. To do so, these chapters lead you through a complete hands-on example, which we recommend following from A to Z in order to ``\textbf{learn by doing}''. Once you have read these two chapters, you will know the crux of Dynare's functionality and (hopefully!) feel comfortable using Dynare for your own work. At that point, though, you will probably find yourself coming back to the User Guide to skim over some of the content in the advanced chapters to iron out details and potential complications you may run into.\\

If, instead, you're an \textbf{intermediate user} of Dynare, you will most likely find the advanced chapters (\ref{ch:soladv} and \ref{ch:estadv}) more appropriate. These chapters cover more advanced features of Dynare and more complicated usage scenarios. The presumption is that you would skip around these chapters, focusing on the topics most applicable to your needs and curiosity. Examples are therefore more concise and specific to each feature; these chapters read a bit more like a reference manual.\\

We also recognize that you probably have had repeated if not active exposure to programming and are likely to have a strong economic background. Thus, a black box solution to your needs is inadequate. To hopefully address this issue, the User Guide goes into some depth in covering the \textbf{theoretical underpinnings and methodologies that Dynare follows} to solve and estimate DSGE models. These are available in the ``Behind the Scenes of Dynare'' chapters (\ref{ch:solbeh} and \ref{ch:estbeh}). These chapters can also serve as a \textbf{basic primer} if you are new to the practice of DSGE model solving and Bayesian estimation. \\

Finally, besides breaking up content into short chapters, we've introduced two different \textbf{markers} throughout the Guide to help streamline your reading.
\begin{itemize}
\item \textbf{\textsf{TIP!}} introduces advice to help you work more efficiently with Dynare or solve common problems.  
\item \textbf{\textsf{NOTE!}} is used to draw your attention to particularly important information you should keep in mind when using Dynare. 
\end{itemize} 


\section{What is Dynare?}
Before we dive into the thick of the ``trees'', let's have a look at the ``forest'' from the top \ldots just what is Dynare? \\

\textbf{Dynare is a powerful and highly customizable engine, with an intuitive front-end interface, to solve, simulate and estimate DSGE models}. \\

In slightly less flowery words, it is a pre-processor and a collection of Matlab routines that has the great advantage of reading DSGE model equations written almost as if they were in an academic paper. This not only facilitates the preparation of a model for processing by Dynare, but also enables you to easily share your code with others as it will be comprehensible by all.\\
\begin{figure}
\begin{center}
\includegraphics[width=1.0\textwidth]{P_DynareStruct2}
\end{center}
\caption[Dynare, a bird's eyeview]{The .mod file being read by the Dynare pre-processor, which then calls the relevant Matlab routines to carry out the desired operations and display the results.}
\label{fig:dyn}
\end{figure}

Figure \ref{fig:dyn} gives you an overview of the way Dynare works. Basically, the model, as well as its related attributes (e.g. shock structure), is written equation by equation in your preferred text editor. The resulting file (called the .mod file) is then called from Matlab. This initiates the Dynare pre-processor, which translates the .mod file into suitable input (i.e. intermediary Matlab or C files) for the Matlab routines used to either solve or estimate the model.  Finally, results are presented in Matlab. Some more details on the internal files generated by Dynare are given in section \ref{sec:dynfiles}. \\

Each of these steps will become clear as you read through the User Guide, but for now it may be helpful to summarize \textbf{what Dynare is able to do}:
\begin{itemize}
\item compute the steady state of a model
\item compute the solution of deterministic models
\item compute the first and second order approximation to solutions of stochastic models
\item estimate parameters of DSGE models using either a maximum likelihood or a Bayesian approach
\item compute optimal policies in linear-quadratic models
\end{itemize}


\section{Additional Sources of Help}
While this User Guide tries to be as complete and thorough as possible, at times you will almost certainly want to browse other material for help. At your disposal, you have the following additional resources:
\begin{itemize}
\item \href{http://www.cepremap.cnrs.fr/juillard/mambo/download/manual/index.html}{\textbf{Reference Manual}}: this manual covers all Dynare commands, giving a clear definition and explanation of usage for each. The User Guide will often introduce you to a command in a rather loose manner (mainly through examples). Thus, reading corresponding command descriptions in the Reference Manual is a good idea to cover all relevant details. 
\item \href{http://www.cepremap.cnrs.fr/juillard/mambo/index.php?option=com_content&task=category&sectionid=11&id=96&Itemid=89}{\textbf{Official online examples}}: the Dynare website includes other (often well-documented) examples of .mod files covering models and methodologies introduced in recent papers. 
\item \href{http://www.cepremap.cnrs.fr/juillard/mambo/index.php?option=com_forum&Itemid=95&page=viewforum&f=2&sid=10290a11eb7a48243971159f5b86f83e}{\textbf{Open online examples}}: this page lists .mod files posted by users covering a wide variety of examples. 
\item \href{http://www.dynare.org/phpBB3}{\textbf{Dynare forums}}: this lively online discussion forum allows you to ask your questions openly and read threads from others who might have run into similar difficulties. 
\item \href{http://www.cepremap.cnrs.fr/juillard/mambo/index.php?option=com_content&task=section&id=3&Itemid=40}{\textbf{Frequently Asked Questions}} (FAQ): this section of the Dynare site emphasizes a few of the most popular questions in the forums. 
\item \href{http://www.dsge.net}{\textbf{DSGE.net}}: this website, run by members of the Dynare team, is a resource for all scholars working in the field of DSGE modeling. Besides allowing you to stay up to date with the most recent papers and possibly make new contacts, it conveniently lists conferences, workshops and seminars that may be of interest. 
\end{itemize}

\section{Nomenclature}
To end this introduction and avoid confusion in what follows, it is worthwhile to \textbf{define a few terms}. Many of these are shared with the Reference Manual. 
\begin{itemize}
\item \textbf{Integer} indicates an integer number.
\item \textbf{Double} indicates a double precision number. The following syntaxes are valid: 1.1e3, 1.1E3, 1.1E-3, 1.1d3, 1.1D3
\item \textbf{Expression} indicates a mathematical expression valid in the underlying language (e.g. Matlab).
\item \textbf{Variable name} indicates a variable name. \textbf{\textsf{NOTE!}} Variable names can be comprised of any combination of alphanumeric characters and underscores, as long as the first character is a letter. All other characters, including accents and \textbf{spaces}, are forbidden. 
\item \textbf{Parameter name} indicates a parameter name which must follow the same naming conventions as above. 
\item \textbf{Filename} indicates a valid file name on your operating system. Note that Matlab requires that names of files or functions begin with a letter; this concerns your Dynare .mod files.
\item \textbf{Command} is an instruction to Dynare or other program when specified. 
\item \textbf{Options} or optional arguments for a command are listed in square brackets \mbox{[ ]} unless otherwise noted. If, for instance, the option must be specified in parenthesis in Dynare, it will show up in the Guide as [(option)].
\item \textbf{\texttt{Typewritten text}} indicates text as it should appear in Dynare code.
\end{itemize}

\section{Dynare Version 4: What's New and Backward Compatibility}
This guide has been written for Dynare version 4. With respect to version 3, this new version introduces several important features, improvements, function optimizations and bug fixes. The major new features are the following: 
\begin{itemize}
\item Analytical derivatives are now used everywhere (for instance, in the Newton algorithm for deterministic models and in the linearizations necessary to solve stochastic models). This increases computational speed significantly. The drawback is that Dynare can now handle only a limited set of functions although, in nearly all economic applications, this should not be a constraint. 
\item Variables and parameters are now kept in the order in which they are declared whenever displayed (and when used internally) by Dynare. Recall that in version 3, variables and parameters were variably displayed in either their order of declaration or in alphabetical order. \textbf{\textsf{NOTE!}} This may cause programs that run off of version 3 output to function incorrectly after switching to Dynare version 4.
\item The names of many internal variables and the organization of output variables has changed. These are enumerated in detail in the relevant chapters. The names of the files internally generated by Dynare have also changed. (** more on this when explaining internal file structure - TBD)
\item The syntax for the external steady state file has changed. See section \ref{sec:findsteady} for more details. \textbf{\textsf{NOTE!}} You will unfortunately have to slightly amend any old steady state files you may have written. 
\item Speed. Several large-scale improvements have been implemented to speed up Dynare. This should be most noticeable when solving deterministic models, but also apparent in other functionality.
\end{itemize}
\chapter{Installing Dynare} \label{ch:inst}

\section{System requirements}
Dynare is supported by Windows 98, 2000, NT and XP; at the time of writing, no experience was available on Vista (** change this). Dynare can also run on Unix (please write to Michel Juillard with questions about support for particular platforms ** is this still the case?) and Mac OS X. \\

To run Dynare, it is recommended to allocate at least 256MB of RAM to the platform running Dynare, although 512MB is preferred. Depending on the type of computations required, like the very processor intensive Metropolis Hastings algorithm, you may need up to 1GB of RAM to obtain acceptable computational times. \\

\section{Installation}
Three versions of Dynare exist: one for Matlab, one for Scilab and one for Gauss. The first benefits from ongoing development and is the most popular. Development of the Scilab version stopped after Dynare version 3.02 and that for Gauss after Dynare version 1.2. \\

This User Guide will focus exclusively on the Matlab version of Dynare. For the installation procedure for the Scilab or Gauss versions of the program, please see the \href{http://www.cepremap.cnrs.fr/juillard/mambo/index.php?option=com_content&task=view&id=51&Itemid=84}{Reference Manual}. Note, though, that the main functionality - and especially syntax - of Dynare remains mostly unchanged across the Matlab, Scilab or Gauss versions, for those features common to the older versions of Dynare. \\

\subsection{Installing Dynare for Matlab on Windows}
The following assumes you have Matlab version 6.5.1 or later installed on your Windows system.
\begin{enumerate}
\item Download the latest stable version of Dynare for Matlab (Windows) from the \href{http://www.cepremap.cnrs.fr/juillard/mambo/index.php?option=com_frontpage&Itemid=1}{Dynare website}. 
\item You will now have on your computer a .zip file which you should unzip. This will create a folder called, by default, Dynare and its version number, for example: Dynare\_v3.0 
\item This directory contains several sub-directories, among which (i) matlab, (ii) doc and (iii) examples. 
\item Place the Dynare folder (Dynare\_v3.0 in our example) in the c: directory and note that location. The easiest is probably to put it in the root of c: as in c:/dynare\_v3.0.
\item Start Matlab and use the menu File/Set-Path to add the path to the Dynare 
matlab subdirectory. Following our example, this would correspond to 
c:/dynare\_v3.0/matlab
\item Save these changes in Matlab and you're ready to go. (** doesn't work if dynare is put in program files, for instance... strange?)
\end{enumerate} 

\subsection{Installing Dynare for Matlab on UNIX}
** TBD - must recompile parser, but need exact instructions. 
\subsection{Installing Dynare for Matlab on Mac OSX}
** TBD - must recompile parser, but need exact instructions. 

\section{Matlab particularities}

A question often comes up: what special Matlab toolboxes are necessary to run Dynare? In fact, no additional toolbox is necessary for running most of Dynare, except maybe for optimal simple rules (see chapter \ref{ch:ramsey}), but even then remedies exist (see the \href{http://www.cepremap.cnrs.fr/juillard/mambo/index.php?option=com_forum&Itemid=95}{Dynare forums} for discussions on this, or ask your particular question there). But if you do have the 'optimization toolbox' installed, you will have additional options for solving for the steady state (solve\_algo option) and for searching for the posterior mode (mode\_compute option), defined later. 



\chapter{Solving DSGE models - basics} \label{ch:solbase}

This chapter focusses on the solution of DSGE models. The term ``solution'' is used rather broadly to encompass a wide spectrum of functionalities and user needs. This chapter should be relevant if you're interested in solving a system of first order and equilibrium conditions for a set of endogenous variables. You may also be interested in how this system behaves in response to shocks, whether expected or stochastic, temporary or permanent. Likewise, you may want to explore how the system comes back to its steady state or moves to a new one. We recommend that you read this chapter chronologically, as its goal is to get you to write a typical Dynare input file and obtain basic results. More advanced features and possible customization are discussed in chapter \ref{ch:soladv}.

\section{Introducing an example}
The goal of this first section is to introduce a simple example right off the bat, in order to give a more tangible context in which to introduce the relevant Dynare features and commands.\\

The model is a simplified standard RBC model taken from \citet{CollardJuillard2003} which served as the original User Guide for Dynare. Other examples along the same lines (from the most basic to more intricate RBC models) with helpful notes and explanations, are provided by Jesus Fernandez-Villaverde in the \href{http://www.cepremap.cnrs.fr/juillard/mambo/index.php?option=com_content&task=category&sectionid=11&id=96&Itemid=89}{official examples} section of the Dynare website. \\

The economy consists of an infinitely living representative agent who values consumption $c_t$ and labor services $h_t$ according to the following utility function
\[
\mathbb{E}_t \sum_{\tau=t}^\infty \beta^{\tau-t} \left( \log (c_t) - \theta \frac{h_t^{1+\psi}}{1+\psi} \right)
\] 
where, as usual, the discount factor $0<\beta<1$, the disutility of labor $\theta > 0$ and the labor supply elasticity $\psi \geq 0$. \\

A social planner maximizes this utility function subject to the budget constraint
\[
c_t + i_t = y_t
\]
where $i_t$ is investment and $y_t$ output. Consumers are therefore also owners of the firms. The economy is a real economy, where part of output can be consumed and part invested to form physical capital. As is standard, the law of motion of capital is given by
\[
k_{t+1} = \exp (b_t)i_t + (1-\delta)k_t
\]
with $0<\delta<1$, where $\delta$ is physical depreciation, $\alpha$ the capital elasticity in the production function and where $b_t$ is a shock affecting incorporated technological progress. \\

We assume output is produced according to a standard constant returns to scale technology of the form
\[
y_t = \exp (a_t)k_t^\alpha h_t^{1-\alpha}
\]
with $0<\alpha<1$, and where $a_t$ represents a stochastic technological shock (or Solow residual). \\

Finally, we specify a \textbf{shock structure} that allows for shocks to display persistence across time and correlation in the current period. That is
\[
\left( \begin{array}{c}a_t \\
b_t \end{array} \right) = \left( \begin{array}{c c} \rho & \tau \\
\tau & \rho \end{array} \right) \left( \begin{array}{c}a_{t-1} \\
b_{t-1} \end{array} \right) + \left( \begin{array}{c}\epsilon_t \\
\nu_t \end{array} \right)
\]
where $|\rho + \tau|<1$ and $|\rho - \tau|<1$ to ensure stationarity (we call $\rho$ the coefficient of persistence and $\tau$ that of cross-persistence). Furthermore, we assume $\mathbb{E}_t (\epsilon_t)=0$, $\mathbb{E}_t (\nu_t)=0$ and that the contemporaneous variance-covariance matrix of the innovations $\epsilon_t$ and $\nu_t$ is given by
\[
\left( \begin{array}{c c} \sigma_\epsilon^2 & \psi \sigma_\epsilon \sigma_\nu \\
\psi \sigma_\epsilon \sigma_\nu & \sigma_\nu^2 \end{array} \right)
\]
and where $corr(\epsilon_t \nu_s)=0$, $corr(\epsilon_t \epsilon_s)=0$ and $corr(\nu_t \nu_s)=0$ for all $t \neq s$. \\

This system - probably quite similar to standard RBC models you have run into - yields the following first order conditions (which are straightforward to reproduce in case you have doubts\ldots) and equilibrium conditions drawn from the description above. Note that the first equation captures the labor supply function and the second the intertemporal consumption Euler equation.
\[
\begin{aligned}
c_t \theta h_t^{1+\psi} = (1-\alpha)y_t \\
1= \beta \mathbb{E}_t \left[ \left( \frac{\exp(b_t)c_t}{\exp(b_{t+1})c_{t+1}} \right) \left( \exp(b_{t+1}) \alpha \frac{y_{t+1}}{k_{t+1}}+1-\delta \right) \right] \\
y_t = \exp (a_t)k_t^\alpha h_t^{1-\alpha} \\
k_{t+1} = \exp (b_t)i_t + (1-\delta)k_t \\
a_t = \rho a_{t-1} + \tau b_{t-1} + \epsilon_t \\
b_t = \tau a_{t-1} + \rho b_{t-1} + \nu_t
\end{aligned}
\] 

Alternatively, for reasons that will become evident later, the system is simple enough that we can use pen and paper to linearize it by hand. This yields the following equations where hats over variables imply percent deviation from steady state and variables without time subscript correspond to steady state values. (** check problems with equations)
\[
\begin{aligned}
\hat{c}_t + (1+\psi) \hat{h}_t = \hat{y}_t \\
\hat{c}_t =  \mathbb{E}_t \hat{c}_{t+1} - (1-\beta + \delta \beta) \mathbb{E}_t \left[ \hat{y}_{t+1} - \hat{k}_{t+1} \right] \\
\hat{y}_t = \alpha \hat{k}_t + (1-\alpha) \hat{h}_t \\
\mathbb{E}_t \hat{k}_{t+1} = \frac{1}{\alpha}\left(\frac{1}{\beta}-1+\delta \right) \hat{y}_t - \frac{c}{k} \hat{c}_t + (1-\delta)\hat{k}_t \\
\hat{a}_t = \rho \hat{a}_{t-1} + \tau \hat{b}_{t-1} + \hat{\epsilon}_t \\
\hat{b}_t = \tau \hat{a}_{t-1} + \rho \hat{b}_{t-1} + \hat{\nu}_t
\end{aligned}
\] 
This ends the exposition of the example. Now, let's roll up our sleeves and see how we can get the model into Dynare and actually generate some impulse response functions. 

\section{Dynare .mod file structure}
Input into Dynare involves the .mod file, as mentioned loosely in the introduction of this Guide. The .mod file can be written in any editor, external or internal to Matlab. It will then be read by Matlab by first navigating within Matlab to the directory where the .mod file is stored and then by typing in the Matlab command line \texttt{Dynare filename.mod;} (although actually typing the extension .mod is not necessary). But before we get into executing a .mod file, let's start by writing one! \\

It is convenient to think of the .mod file as containing four distinct parts, illustrated in figure \ref{fig:modstruct}:
\begin{itemize}
\item preamble: lists variables and parameters
\item model: spells out the model
\item dynamics: gives indications to find the steady state and defines the shocks to the system
\item computation: asks to undertake specific operations (e.g. forecasting, estimating impulse response functions)
\end{itemize}
Our exposition below will follow these distinctions. 
\begin{figure} \label{fig:modstruct}
\begin{center} 
\includegraphics[width=1.0\textwidth]{P_ModStruct2} 
\end{center} 
\caption[Structure of the .mod file]{The .mod file contains four logically distinct parts.} 
\end{figure}

\section{Filling out the preamble}
The preamble generally involves three commands that tell Dynare what are the model's variables, which are endogenous and what are the parameters. The \textbf{commands} are:
\begin{itemize}
\item \texttt{var} starts the list of endogenous variables, to be separated by commas. 
\item \texttt{varexo} starts the list of exogenous variables that will be shocked.
\item \texttt{parameters} starts the list of parameters and assigns values to each.
\end{itemize}

In the case of our example, here's what the \textbf{preamble would look like}:\\ 
\\
\texttt{var y, c, k, a, h, b;\\
varexo e,u;\\
parameters beta, rho, beta, alpha, delta, theta, psi, tau;\\
alpha = 0.36;\\
rho   = 0.95;\\
tau   = 0.025;\\
beta  = 0.99;\\
delta = 0.025;\\
psi   = 0;\\
theta = 2.95;\\
phi   = 0.1;}\\

As you can tell, writing a .mod file is really quite straightforward. Two quick comments: \textsf{\textbf{TIP!}} Remember that each line of the .mod file must be terminated by a semicolon (;). To comment out any line, start the line with two forward slashes (//).  

\section{Specifying the model} \label{sec:modspe}
\subsection{Model in Dynare notation}
One of the beauties of Dynare is that you can input your model's equations almost as if you were writing them to communicate them to a person - not a machine. No fancy transformations, no matrix manipulations, no ambiguous deliberations about the backward or forward looking nature of variables are necessary. Dynare's pre-processor does all the grunt work to build and populate the various matrices necessary to untangle and solve the system. Let's first have a look at our model in Dynare notation, and then take a moment to point out some of its important features. \\
\\
\texttt{model;\\
c*theta*h$\hat{ }$(1+psi)=(1-alpha)*y;\\
k = beta*(((exp(b)*c)/(exp(b(+1))*c(+1)))\\
    *(exp(b(+1))*alpha*y(+1)+(1-delta)*k));\\
y = exp(a)*(k(-1)$\hat{ }$alpha)*(h$\hat{ }$(1-alpha));\\
k = exp(b)*(y-c)+(1-delta)*k(-1);\\
a = rho*a(-1)+tau*b(-1) + e;\\
b = tau*a(-1)+rho*b(-1) + u;\\
end;}\\

\subsection{Some commands and conventions}
The above example illustrates the use of a few important commands and conventions to translate a model into a Dynare-readable .mod file. 
\begin{itemize}
\item The first thing to notice, is that the model block of the .mod file begins with the command \texttt{model} and ends with the command \texttt{end}. 
\item Second, in between, there need to be as many equations as you declared endogenous parameters (this is actually one of the first things that Dynare checks; it will immediately let you know if there are any problems). 
\item Third, as in the preamble and everywhere along the .mod file, each line ends with a semicolon (except when a line is too long and you want to break it across two lines). 
\item Fourth, equations are entered one by one, with the same variable and parameter names as declared in the preamble. 
\item Fifth, the specification of the equations looks almost identical to those listed as the first order conditions above, except for a few important differences. These make up the variable input conventions and are listed in the box below for greater clarity.
\end{itemize}

\begin{tabular}{|p{11cm}|}
\hline
\textbf{Variable input conventions}\\
\begin{enumerate}
\item Variables entering the system with a time $t$ subscript are written plainly. For example, $x_t$ would be written $x$. 
\item Variables entering the system with a time $t-n$ subscript are written with $(-n)$ following them. For example, $x_{t-2}$ would be written $x(-2)$ (incidentally, this would count as two backward looking variables). 
\item In the same way, variables entering the system with a time $t+n$ subscript are written with $(+n)$ following them. For example, $x_{t+2}$ would be written $x(+2)$.
\item The only slight complications has to do with stock variables. One rule that may help is: in Dynare, the timing of each variable reflects when that variable is decided. For instance, our capital stock today is not decided today, but yesterday (recall, it is a function of yesterday's investment and capital stock). Thus, we would write it with a lag, meaning that the $k_{t+1}$ in our model equations gets reported into the .mod file as $k$ (at time $t$).
\item A slightly more roundabout way to explain the same thing is that for stock variables, you must use a ``stock at the end of the period'' concept. It is investment during period $t$ that sets stock at the end of period $t$. Be careful, a lot of papers use the ``stock at the beginning of the period'' convention. For instance, in some wage negociation models, wages used during a period are set the period before. In the equation for wages, you can write wage in period $t$ (when they are set), but in the labor demand equation, wages should appear with a one period lag. 
\end{enumerate}
\\
\hline
\end{tabular}\\
\\

\subsection{Predetermined vs. non-predetermined variables}
What does all this mean for predetermined vs. non-predetermined variables? If you are used to solving simple DSGE models by hand, or to inputting the matrices corresponding to your equations, you are probably eagerly waiting for an answer. \textsf{\textbf{TIP!}} Actually, you may surprised (relieved?!) to know that Dynare figures out automatically which variables are predetermined and non-predetermined. Especially in more complicated models, the categorization of variables can be difficult, if not ambiguous, and a source of much frustration.\\ 

In Dynare, a forward-looking variable is a variable that appears in the model with a lead. For example, $c(+1)$.  A predetermined variable is a variable that appears with a lag. Note that a variable can be both forward-looking and predetermined. A static variable is a variable that appears in the model with neither a lead or a lag.\\

\subsection{Linear and log-linearized models}
There are two other variants of the system's equations which Dynare accommodates. First, the \textbf{linear model} and second, the \textbf{model in exp-logs}. In the first case, all that is necessary is to write the term \texttt{(linear)} next to the command \texttt{model}. In our example, this would yield the following for the first two lines:\\

** complete once the fixed problem with linearized equations\\

Otherwise, you may be interested to have Dynare take Taylor series expansions in logs rather than in levels. If so, simply rewrite your equations by taking $\exp$ and $\log$ of each variable. The Dynare input convention makes this very easy to do. Our example would need to be re-written as:\\
\\
\texttt{model;\\
exp(c)*theta*exp(h)$\hat{ }$(1+psi)=(1-alpha)*exp(y);\\
exp(k) = beta*(((exp(b)*exp(c))/(exp(b(+1))*exp(c(+1))))\\
         *(exp(b(+1))*alpha*exp(y(+1))+(1-delta)*exp(k)));\\
exp(y) = exp(a)*(exp(k(-1))$\hat{ }$alpha)*(exp(h)$\hat{ }$(1-alpha));\\
exp(k) = exp(b)*(exp(y)-exp(c))+(1-delta)*exp(k(-1));\\
a = rho*a(-1)+tau*b(-1) + e;\\
b = tau*a(-1)+rho*b(-1) + u;\\
end;}\\
so that the levels of variables are given by $\exp(variable)$.\\

\subsection{Taking advantage of Matlab}
(\textbf{\textsf{TIP!}}) Dynare accepts standard Matlab expressions in any of the model equations. The only thing you need to do is to start the line with \# and then enter the Matlab command as you would normally. For example, if you want to define a variable in terms of another, you can write \texttt{\# b = 1 / c}. This feature, actually, is more useful when carrying out model estimation and wanting to declare transformations of parameters. 

\section{Specifying dynamics (steady states and shocks)}
In the ``dynamics'' block of the .mod file, you need to specify two things: steady state values and shocks to the exogenous variables; basically, where is your model starting from (steady state, usually) and where is it going (due to the shocks)? Let's briefly describe the relevant commands and then worry about how to mix and match them depending on the type of shock you want to create and the type of model you're dealing with.\\

\subsection{Specifying steady states}
To specify steady states, you use the commands \texttt{initval}, \texttt{endval} or, more rarely, \texttt{histval}. Each command initiates a block which must be terminated by \texttt{end}. In the middle, you should list values for all your variables - endogenous and exogenous (even if the latter is not required for stochastic models, but you may want to get in the habit of doing so anyhow). The \texttt{initval} command allows you to enter your variables' steady state values before any shocks hit the system. Under the \texttt{endval} command, instead, you would list the steady state values to which the system moves after any shocks (assuming the system does not return to its original steady state). Finally, the \texttt{histval} command allows you to specify different historical initial values for different periods, in case you have lagged variables. But because this last command is rarely used, we will not dwell on it; you can instead read more about it in the Reference Manual (** add link). We now focus on the \texttt{intival} block, which is common to all .mod files, and then return later to the \texttt{endval} block, which should be used only in certain circumstances. \\

For our example, here is what the \texttt{initval} block would look like:\\
\\
\texttt{initval;\\
y = 1.08068253095672;\\
c = 0.80359242014163;\\
h = 0.29175631001732;\\
k = 5;\\
a = 0;\\
b = 0;\\
e = 0;\\
u = 0;\\
end;}\\

The difficulty here, of course, is calculating the actual steady state values. Because this is such a central topic, although it borders on a form of art, rather than a science, we've devoted the box below to some \textsf{\textbf{TIPS!}} regarding the calculation of steady states.\\

\begin{tabular}{|p{11cm}|}
\hline
\textbf{Calculating steady states}\\
\\
Suppose, first, that you were interested in entering the exact values of your model's steady states. You could, most straightforwardly, use an external program to calculate these values, like Maple and then enter them by hand. But of course, this procedure could be time consuming and bothersome, especially if you want to alter parameter values. \\

Actually, Dynare can take care of finding your model's steady state by calling the appropriate Matlab functions. To do so, you still need to provide an initial guess of your steady state in the \texttt{initval} block, but you would follow the block with the command \texttt{steady[ \textsl{option} ]}. This will instruct Matlab to use an iterative procedure to find the model's steady state. The procedure itself depends on the option you specify. Possible options are:\texttt{solve\_algo = 0}: uses Matlab Optimization Toolbox FSOLVE; \texttt{solve\_algo = 1}: uses Dynare�s own nonlinear equation solver; and \texttt{solve\_algo = 2}: splits the model into recursive blocks and solves each block in turn. The latter is the default option if none is specified. \\

For complicated models, finding good numerical initial values for the endogenous variables is the trickiest part of finding the equilibrium of that model. Often, it is better to start with a smaller model and add new variables one by one.\\

Unfortunately, for simpler models, you may still run into difficulties in finding your steady state. If so, one option is still available: that of entering your model in linear terms (where variables are expressed in percent deviations from steady state) and thus specifying your steady state values in the \texttt{initval} block as all being zero. \\
\\
\hline
\end{tabular}\\
\\

\subsection{Deterministic vs. stochastic models}
Before we specify shocks, we must decide if our model is deterministic or stochastic. The distinction has actually been a source of significant confusion in the past, so let's try to come to terms with it. The following box defines each type of model:

\begin{tabular}{|p{11cm}|}
\hline
\textbf{Deterministic vs stochastic models}\\
\\
Deterministic models have the following characteristics:
\begin{enumerate}
\item In the recent DSGE literature, deterministic models are somewhat rare. Examples include OLG models without aggregate uncertainty.
\item Models assume full information, perfect foresight, no uncertainty around shocks.
\item Shocks can hit the economy today or at any time in the future, in which case they would be expected with perfect foresight. They can also last one or several periods.
\item Most often, though, models introduce a positive shock today and zero shocks thereafter (with certainty).
\item The solution does not require linearization, as the paths of each variable are fully known.
\item This solution method can therefore be useful when the economy is far away from steady state (when linearizations can provide highly distorted results).
\end{enumerate}\\
Stochastic models, instead, have the following characteristics:
\begin{enumerate}
\item These types of models tend to be more popular in the literature. Examples include most RBC models, or new keynesian monetary models.
\item In these models, shocks hit today (with a surprise), but thereafter their expected value is zero. Thus, these models do not allow for future, expected shocks, nor permanent changes in the exogenous variables.
\item Note that when these models are linearized to the first order, the expectation of the shocks being zero translates into the shocks actually being zero, by certainty equivalence. This is an often overlooked point in the literature which misleads readers in supposing their models may be deterministic.
\item Solutions are technically not simulations (although the literature commonly refers to impulse response functions as being �simulated�). They are instead �solutions� to a constrained maximization problem.
\end{enumerate}
\\
\hline
\end{tabular}\\
\\

\subsection{Adding shocks}
Now let's see how this section's relevant commands - specifying initial and final values, as well as shocks - all fit together depending on the type of model you are using. Figure \ref{fig:shockmodel} summarizes the possible combinations of commands and their usefulness. 
\begin{figure} \label{fig:shockmodel}
\begin{center} 
\includegraphics[width=1.0\textwidth]{P_ShockModel2} 
\end{center} 
\caption[Shock and model-type matrix]{Depending on the model type you're working with and the desired shocks, you will need to mix and match the various steady state and shock commands.} 
\end{figure}\\

As seen in the figure, when working with a deterministic model, you have the choice of introducing both temporary and permanent shocks. In the \textbf{deterministic-temporary} case, you are free to set the duration and level of the shock. To specify a shock that lasts 9 periods on $\epsilon_t$, for instance, you would write (** is this a 10\% shock?)\\
\\
\texttt{shocks;\\
var e;
periods 1:9;\\
values 0.1;\\
end;}\\

In the \textbf{deterministic-permanent} case, you would not specify actual ``shocks'', but would simply tell the system to which steady state values you would like it to move to and let Dynare calculate the transition path. To do so, you would use the \texttt{endval} block following the usual \texttt{initval} block. For instance, you may specify all values to remain common between the two blocks, except for the value of the exogenous variable that changes permanently. The example we have used so far is somewhat ill-suited to illustrate this sort of permanent change, since if the exogenous shocks were positive always, they would make the technology variables $a_t$ and $b_t$ explode. Thus, for illustrative purposes, we could imagine getting rid of the last two equations of our system (the equations of motion of the technology variables) as well as the variables $\epsilon_t$ and $\nu_t$, and simply making $a_t$ and $b_t$ exogenous. In that case, we could use the following blocks in our .mod file to introduce a permanent increase in productivity: \\
\\
\texttt{initval;\\
y = 1.08068253095672;\\
c = 0.80359242014163;\\
h = 0.29175631001732;\\
k = 5;\\
a = 0;\\
b = 0;\\
end;\\
\\
endval;\\
y = 1.08068253095672;\\
c = 0.80359242014163;\\
h = 0.29175631001732;\\
k = 5;\\
a = 0.1;\\
b = 0;\\
end;\\
\\
steady;}\\
\\
where the last command \texttt{steady} was added since the \texttt{endval} values are bound to be somewhat imprecise now that we have changed the value of $a_t$. \\

In case of a \textbf{stochastic-temporary} model, which is really what makes most sense for the example introduced earlier, we can only specify a temporary shock lasting one period. In the case of our example, we would write:\\
\\
\texttt{shocks;\\
var e; stderr 0.009;\\
var u; stderr 0.009;\\
var e, u = phi*0.009*0.009;\\
end;}\\
\\
where the last line specifies the contemporaneous correlation between our two exogenous variables. Note that to introduce the variance of the shock, we could also have written: (** does this mean every time you run your .mod file you will get a slightly different shock?)\\
\\
\texttt{shocks;\\
var e = 0.009 $\hat{}$ 2;\\
var u = 0.009 $\hat{}$ 2;\\
var e, u = phi*0.009*0.009;\\
end;}\\

\subsection{Some useful details}
Two last things to note: first, you can actually \textbf{mix in deterministic shocks} in stochastic models by using the commands \texttt{varexo\_det} and listing some shocks as lasting more than one period in the \texttt{shocks} block. For information on how to do so, please see the Reference Manual (** add link).\\

Second, a handy command that you can add after the \texttt{initval} or \texttt{endval} block (following the \texttt{steady} command if you decide to add on) is the \texttt{check} command. This computes and displays the eigenvalues of your system which are used in the solution method. A necessary condition for the uniqueness of a stable equilibrium in the neighborhood of the steady state is that there are 
as many eigenvalues larger than one in modulus as there are forward looking variables in the system. If this condition is not met, Dynare will tell you that the Blanchard-Kahn conditions are not satisfied (whether or not you insert the \texttt{check} command). \\

One thing to note: don't worry if you get infinite eigenvalues - these are are firmly grounded in the theory of generalized eigenvalues. They have no detrimental influence on the solution algorithm. As far as Blanchard-Kahn conditions are concerned infinite eigenvalues counted as explosive roots of modulus larger than one. \\

\subsection{The complete dynamics of the .mod file}
Let's bring together all the disparate parts discussed above in order to get a complete view of what the "dynamics" part of the .mod file looks like for our ``stochastic-temporary shock'' type model (we've seen only bits and pieces of it so far...).\\
\\
\texttt{initval;\\
y = 1.08068253095672;\\
c = 0.80359242014163;\\
h = 0.29175631001732;\\
k = 5;\\
a = 0;\\
b = 0;\\
e = 0;\\
u = 0;\\
end;\\
\\
shocks;\\
var e; stderr 0.009;\\
var u; stderr 0.009;\\
var e, u = phi*0.009*0.009;\\
end;}\\

\section{Selecting a computation} \label{sec:compute}
So far, we have written an instructive .mod file, but what should Dynare do with it? What are we interested in? In most cases, it will be impulse response functions (IRFs) due to the external shocks. Let's see which are the appropriate commands to give to Dynare.\\

\subsection{For deterministic models}
In the deterministic case, all you need to do is add the command \texttt{simul} at the bottom of your .mod file. Note that the command takes the option \texttt{[ (periods=INTEGER) ] }. \texttt{Simul} triggers the computation of a deterministic simulation of the model for the number of periods set in the option. To do so, it uses a Newton method to solve simultaneously all the equations for every period (see \citet{Juillard1996} for details). \textbf{The simulated variables} are available in global matrix y\_ in which variables are arranged row by row, in alphabetical order. \\

\subsection{For stochastic models}
In the more popular case of a stochastic model, as in our example, the command \texttt{stoch\_simul} is appropriate. This command instructs Dynare to compute a Taylor approximation of the decision and transition functions for the model (the equations listing current values of the endogenous variables of the model as a function of the previous state of the model and current shocks), impulse response 
functions and various descriptive statistics (moments, variance decomposition, correlation and autocorrelation coeffi- 
cients).\footnote{For correlated shocks, the variance decomposition is computed as in the VAR literature through a Cholesky 
decomposition of the covariance matrix of the exogenous variables. When the shocks are correlated, the variance 
decomposition depends upon the order of the variables in the varexo command. (** not sure I understand this).} If you're interested to peer a little further into what exactly is going on behind the scenes of Dynare's computations, have a look at Chapter \ref{ch:soladv}. Here instead, we focus on the application of the command and reproduce below the table of options that can be added to \texttt{stoch\_simul}. (** maybe just include a subset and refer to Ref Man for complete set?) \\

\textbf{Options following the \texttt{stoch\_simul} command:}\\
\begin{itemize}
\item ar = INTEGER: Order of autocorrelation coefficients to compute and to print (default = 5).
\item dr\_algo = 0 or 1: specifies the algorithm used for computing the quadratic approximation of the decision rules: $0$ uses a pure perturbation approach as in \citet{SchmittGrohe2004} (default) and 1 moves the point around which the Taylor expansion is computed toward the means of the distribution as in \citet{CollardJuillard2001a}.
\item drop = INTEGER: number of points dropped at the beginning of simulation before computing the summary 
statistics (default = 100).
\item hp\_filter = INTEGER: uses HP filter with lambda = INTEGER before computing moments (default: no filter).
\item hp\_ngrid = INTEGER: number of points in the grid for the discreet Inverse Fast Fourier Transform used in the 
HP filter computation. It may be necessary to increase it for highly autocorrelated processes (default = 512).
\item irf = INTEGER: number of periods on which to compute the IRFs (default = 40). Setting IRF=0, suppresses the 
plotting of IRF�s. 
\item relative\_irf requests the computation of normalized IRFs in percentage of the standard error of each shock.
\item nocorr: doesn�t print the correlation matrix (printing is the default).
\item nofunctions: doesn�t print the coefficients of the approximated solution (printing is the default). 
\item nomoments: doesn�t print moments of the endogenous variables (printing them is the default).
\item noprint: cancel any printing; usefull for loops. 
\item order = 1 or 2 : order of Taylor approximation (default = 2), unless you're working with a linear model in which case the order is automatically set to 1.  
\item periods = INTEGER: specifies the number of periods to use in simulations (default = 0). 
\item qz\_criterium = INTEGER or DOUBLE: value used to split stable from unstable eigenvalues in reordering the 
Generalized Schur decomposition used for solving 1st order problems (default = 1.000001).
\item replic = INTEGER: number of simulated series used to compute the IRFs (default = 1 if order = 1, and 50 
otherwise).
\item simul\_seed = INTEGER or DOUBLE or (EXPRESSION): specifies a seed for the random generator so as to obtain the 
same random sample at each run of the program. Otherwise a different sample is used for each run (default: seed 
not specified). (** not sure I understand this)
\end{itemize}

Going back to our good old example, suppose we are interested in printing all the various measures of moments of our variables, want to see impulse response functions for all variables, are basically happy with all default options and want to carry out simulations over a good number of periods. We would then end our .mod file with the following command:\\
\\
\texttt{stoch\_simul(periods=2100);}\\

\section{The complete .mod file}
For completion's sake, and for the pleasure of seeing our work bear its fruits, here is the complete .mod file corresponding to our example:\\
\\
\texttt{var y, c, k, a, h, b;\\
varexo e,u;\\
parameters beta, rho, beta, alpha, delta, theta, psi, tau;\\
alpha = 0.36;\\
rho   = 0.95;\\
tau   = 0.025;\\
beta  = 0.99;\\
delta = 0.025;\\
psi   = 0;\\
theta = 2.95;\\
phi   = 0.1;\\
\\
model;\\
c*theta*h$\hat{ }$(1+psi)=(1-alpha)*y;\\
k = beta*(((exp(b)*c)/(exp(b(+1))*c(+1)))\\
    *(exp(b(+1))*alpha*y(+1)+(1-delta)*k));\\
y = exp(a)*(k(-1)$\hat{ }$alpha)*(h$\hat{ }$(1-alpha));\\
k = exp(b)*(y-c)+(1-delta)*k(-1);\\
a = rho*a(-1)+tau*b(-1) + e;\\
b = tau*a(-1)+rho*b(-1) + u;\\
end;\\
\\
initval;\\
y = 1.08068253095672;\\
c = 0.80359242014163;\\
h = 0.29175631001732;\\
k = 5;\\
a = 0;\\
b = 0;\\
e = 0;\\
u = 0;\\
end;\\
\\
shocks;\\
var e; stderr 0.009;\\
var u; stderr 0.009;\\
var e, u = phi*0.009*0.009;\\
end;\\
\\
stoch\_simul(periods=2100);}\\

As per our earlier discussion, we could have added \texttt{steady;} and \texttt{check;} after the \texttt{initval\ldots end;} block.

\section{Executing the .mod file and interpreting results}
To see this all come to life, run this .mod file, which is conveniently installed by default in the Dynare ``examples'' directory (the .mod file is called example1.mod). To run a .mod file, navigate within Matlab to the directory where the example1.mod is stored. You can do this by clicking in the ``current directory'' window of Matlab, or typing the path directly in the top white field of Matlab. Once there, all you need to do is place your cursor in the Matlab command window and type \texttt{dynare example1;} to execute your .mod file. \\

Running example1.mod should take at most 30 seconds. As a result, you should get two forms of output - tabular in the Matlab command window and graphical in two pop-up windows. The tabular results can be summarized as follows:
\begin{enumerate}
\item \textbf{Model summary:} a count of the various variable types in your model (endogenous, jumpers, etc...).
\item \textbf{Matrix of covariance of exogenous shocks:} this should square with the values of the shock variances and co-variances you provided in the .mod file.
\item \textbf{Policy and transition functions:} Solving the rational exectation model, $\mathbb{E}_t[f(y_{t+1},y_t,y_{t-1},u_t)]=0$ , means finding an unkown function, $y_t = g(y_{t-1},u_t)$  that could be plugged into the original model and satisfy the implied restrictions (the first order conditions). A first order approximation of this function can be written as $y_t = \bar{y} + g_y \hat{y}_{t-1} + g_u u_t$, with $\hat{y}_t = y_t-\bar{y}$ and $\bar{y}$ being the steadystate value of $y$, and where $g_x$ is the partial derivative of the $g$ function with respect to variable $x$. In other words, the function $g$ is a time recursive (approximated) representation of the model that can generate timeseries that will approximatively satisfy the rational expectation hypothesis contained in the original model. In Dynare, the table ``Policy and Transition function'' contains the elements of $g_y$ and $g_u$. Details on the policy and transition function can be found in Chapter \ref{ch:estadv}.
\item \textbf{Moments of simulated variables:} up to the fourth moments.
\item \textbf{Correlation of simulated variables:} these are the contemporaneous correlations, presented in a table.
\item \textbf{Autocorrelation of simulated variables:} up to the fifth lag, as specified in the options of \texttt{stoch\_simul}. 
\end{enumerate}

** Add graphical results (last minute omission, my mistake...)










\chapter{Solving DSGE models - advanced topics} \label{ch:soladv}

This chapter is a collection of topics - not all related to each other - that you will probably find interesting or at least understandable, if you have read the earlier Chapter \ref{ch:solbase} on the basics of solving DSGE models. This chapter begins with a section covering the methodologies used by Dynare to solve DSGE models. It then includes a section on undertaking more advanced or custom features in Dynare, like running loops to compare impulse response functions on the same graph, or retrieving particular series of data from your output files. Finally, the chapter ends with a section on modeling tips optimized for Dynare, but possibly also helpful for other work.\\

\section{Behind the scenes: an intro to solving stochastic DSGE models}
The aim of this section is to peer behind the scenes of Dynare, or under its hood, to get an idea of the methodologies and algorithms used in its computations. Going into details would be beyond the scope of this User Guide which will instead remain at a high level. What you will find below will either comfort you in realizing that Dynare does what you expected of it - and what you would have also done if you had had to code it all yourself (with ten extra years of life expectancy and a direct phone line to the Dynare development team!), or will spur your curiosity to have a look at more detailed material. If so, you may want to go through Michel Juillard's presentation on solving DSGE models to a first and second order (** add link), or read \citet{CollardJuillard2001b} or \citet{SchmittGrohe2004} which gives a good overview of the most recent techniques based on perturbation methods. \\

\subsection{What is the advantage of a second order approximation?}
As noted in chapter \ref{ch:solbase} and as will become clear in the section below, linearizing a system of equations to the first order raises the issue of certainty equivalence. This is because only the first moments of the shocks enter the linearized equations, and when expectations are taken, they disappear. Thus, unconditional expectations of the endogenous variables are equal to their non-stochastic steady state values. \\

This may be an acceptable simplification to make. But depending on the context, it may instead be quite misleading. For instance, when using second order welfare functions to compare policies, you also need second order approximations of the policy function. Yet more clearly, in the case of asset pricing models, linearizing to the second order enables you to take risk (or the variance of shocks) into consideration - a highly desirable modeling feature. It is therefore very convenient that Dynare allows you to choose between a first or second order linearization of your model in the option of the \texttt{stoch\_simul} command. \\

\subsection{How does dynare solve stochastic DSGE models?}
In this section, we shall briefly overview the perturbation methods employed by Dynare to solve DSGE models to a first order approximation. The second order follows very much the same approach, although at a higher level of complexity. The summary below is taken in most part from Michel Juillard's presentation ``Computing first order approximations of DSGE models with Dynare'' (** add link), which you should read if interested in particular details. \\

To summarize, a DSGE model is a collection of first order and equilibrium conditions that take the general form: 
\[
\mathbb{E}_t\left\{f(y_{t+1},y_t,y_{t-1},u_t)\right\}=0
\]
\begin{eqnarray*}
\mathbb{E}(u_t) &=& 0\\
\mathbb{E}(u_t u_t') &=& \Sigma_u
\end{eqnarray*}
and where:
\begin{description}
  \item[$y$]: vector of endogenous variables
  \item[$u$]: vector of exogenous stochastic shocks
\end{description}

The solution to this system is a set of equations relating variables in the current period to the past state of the system and current shocks, that satisfy the original system. This is what we call the policy function. Sticking to the above notation, we can this function as:
\[
y_t = g(y_{t-1},u_t)
\]

Then, it is straightforward to re-write $y_{t+1}$ as
\begin{eqnarray*}
  y_{t+1} &=& g(y_t,u_{t+1})\\
  &=& g(g(y_{t-1},u_t),u_{t+1})\\
\end{eqnarray*}

We can then define a new function $F$, such that:
\[
F(y_{t-1},u_t,u_{t+1}) =
f(g(g(y_{t-1},u_t),u_{t+1}),g(y_{t-1},u_t),y_{t-1},u_t)\\
\]
which enables us to rewrite our system in terms of past variables, and current and future shocks:
\[
\mathbb{E}_t\left[F(y_{t-1},u_t,u_{t+1})\right] = 0
\]

We then venture to linearize this model around a steady state defined as:
\[
f(\bar y, \bar y, \bar y) = 0
\]
having the property that:
\[
\bar y = g(\bar y, 0)
\]

The first order Taylor expansion around $\bar y$ yields:
\begin{eqnarray*}
\lefteqn{\mathbb{E}_t\left\{F^{(1)}(y_{t-1},u_t,u_{t+1})\right\} =}\\
&& \mathbb{E}_t\Big[f(\bar y, \bar y, \bar y)+f_{y_+}\left(g_y\left(g_y\hat y+g_uu \right)+g_u u' \right)\\
&& + f_{y_0}\left(g_y\hat y+g_uu \right)+f_{y_-}\hat y+f_u u\Big]\\
&& = 0
\end{eqnarray*}
with $\hat y = y_{t-1} - \bar y$, $u=u_t$, $u'=u_{t+1}$, $f_{y_+}=\frac{\partial f}{\partial y_{t+1}}$, $f_{y_0}=\frac{\partial f}{\partial y_t}$, $f_{y_-}=\frac{\partial f}{\partial y_{t-1}}$, $f_{u}=\frac{\partial f}{\partial u_t}$, $g_y=\frac{\partial g}{\partial y_{t-1}}$, $g_u=\frac{\partial g}{\partial u_t}$.\\

Taking expectations (we're almost there!):
\begin{eqnarray*}
   \lefteqn{\mathbb{E}_t\left\{F^{(1)}(y_{t-1},u_t, u_{t+1})\right\} =}\\
&& f(\bar y, \bar y, \bar y)+f_{y_+}\left(g_y\left(g_y\hat y+g_uu \right) \right)\\
&& + f_{y_0}\left(g_y\hat y+g_uu \right)+f_{y_-}\hat y+f_u u\Big\}\\
&=& \left(f_{y_+}g_yg_y+f_{y_0}g_y+f_{y_-}\right)\hat y+\left(f_{y_+}g_yg_u+f_{y_0}g_u+f_{u}\right)u\\
&=& 0\\
\end{eqnarray*}

As you can see, since future shocks only enter with their first moments (which are zero in expectations), they drop out when taking expectations of the linearized equations. This is technically why certainty equivalence holds in a system linearized to its first order. The second thing to note is that we have two unknown variables in the above equation: $g_y$ and $g_u$ each of which will help us recover the policy function $g$. \\

Since the above equation is true if the first and second parentheses are each zero, we can take each at a time (** hmm?). The first, yields a quadratic equation  in $g_y$, which we can solve with a series of algebraic trics that are not all immediately apparent (but detailed in Michel Juillard's presentation). Incidentally, one of the conditions that comes out of the solution of this equation is the Blanchard-Kahn condition: there must be as many roots larger than one in modulus as there are forward-looking variables in the model. Having recovered $g_y$, recovering $g_u$ is then straightforward from the second parenthesis. \\

Finally, notice that a first order linearization of the function $g$ yields:
\[
y_t = \bar y+g_y\hat y+g_u u
\]
And now that we have $g_y$ and $g_u$, we have solved for the (approximate) policy (or decision) function and have succeeded in solving our DSGE model. If we were interested in impulse response functions, for instance, we would simply iterate the policy function starting from an initial value given by the steady state. \\

The second order solution uses the same ``perturbation methods'' as above (the notion of starting from a function you can solve - like a steady state - and iterating forward), yet applies more complex algebraic techniques to recover the various partial derivatives of the policy function. But the general approach is perfectly isomorphic. Note that in the case of a second order approximation of a DSGE model, the variance of future shocks remains after taking expectations of the linearized equations and therefore affects the level of the resulting policy function.\\

\section{Advanced topics and functions in Dynare}
This section lists some of the more popular customizations of Dynare or advanced features that users may be interested in. 

\subsection{Finding and saving your output}
Where is output stored? ** TBD, add a table summarizing what results exist and where they are stored in version 4. \\

To save the data after simulations, you can just type: \texttt{save $<$datafilename$>$ $<$list of observed variables$>$} at the end of your .mod file. (** should the variables have the same name as in the .mod file?)

\subsection{Using loops to compare IRFs on the same graph}
** See irf\_loop.mod e.g.

\subsection{Referring to external files}
You may find it convenient to refer to an external file, either when computing the steady state of a model in Matlab and reading the values into a .mod file, or when specifying shocks in an external file. The former used to be convenient in Dynare version 3, but is no longer necessary in version 4; we therefore do not provide details. (** give some details anyhow? describe how version 4 deals with steady states of non-stationary variables?) But you may find the latter helpful, for instance, to simulate a model based on shocks from a prior estimation. You could then retrieve the exogenous shocks from the oo\_ file by saving them in a file called datafile.mat. Finally, you could simulate a deterministic model with the shocks saved from the estimation by specifying the source file for the shocks, using the \texttt{shocks(shocks\_file = datafile.mat)} command. (** how should the variables be ordered in this datafile?)
On the other hand, you could use the built-in commands in Dynare to generate impulse response functions from estimated shocks, as described in Chapter \ref{ch:estbase}. \\

\section{Modeling tips}

\subsection{Expectations taken in the past}
For instance, to enter the term $\mathbb{E}_{t-1}y_t$, define $s_t=\mathbb{E}_t[y_{t+1}]$ and then use $s(-1)$ in your .mod file.

\subsection{Stationarizing your model}
Models in Dynare must be stationary, such that you can linearize them around a steady state and return to steady state after a shock. Thus, you must first stationarize your model, then linearize it, either by hand, or by letting Dynare do the work.  You can then reconstruct ex-post the non-stationary simulated variables after running impulse response functions.\\

For deterministic models, the trick is to use only stationary variables in $t+1$. More generally, if $y_t$ is $I(1)$, you can always write $y_{t+1}$ as $y_t+dy_{t+1}$, where $dy_t= y_t-y_{t-1}$. Of course, you need to know the value of $dy_t$ at the final equilibrium.\\

Note that in a stationary model, it is expected that variables will eventually go back to steady state after the initial shock. If you expect to see a growing curve for a variable, you are thinking about a growth model. Because growth models are nonstationary, it is easier to work with the stationarized version of such models. Again, if you know the trend, you can always add it back after the simulation of the stationary components of the variables. 

\subsection{Infinite sums}
Dynare cannot deal with infinite sums very well. For instance, a user was once interested in entering 
\[
y_t=\sum_{j=0}^{\infty} \mathbb{E}_{t-j}x_t
\]
where $y_t$ and $x_t$ are endogenous variables. \\

In Dynare, the best way to handle this is to write out the first $k$ terms explicitly and enter each one in Dynare, such as: $\mathbb{E}_{t-1}x_t + \mathbb{E}_{t-2}x_t+\ldots + \mathbb{E}_{t-k}x_t$.

\chapter{Estimating DSGE models - basics} \label{ch:estbase}

As in the chapter \ref{ch:solbase}, this chapter is structured around an example. The goal of this chapter is to lead you through the basic functionality in Dynare to estimate models using Bayesian techniques, so that by the end of the chapter you should have the capacity to estimate a model of your own. This chapter is therefore very practically-oriented and abstracts from the underlying computations that Dynare undertakes to estimate a model; that subject is instead covered in some depth in Chapter \ref{ch:estadv} on advanced topics. \\

\section{Introducing the example} \label{sec:modsetup}
The example that we will follow in some details in this chapter is drawn from \citet{Schorfheide2000}. This first section introduces the model, its basic intuitions and equations. We will then see in subsequent sections how to estimate it using Dynare. Note that the original paper by Schorfheide mainly focusses on estimation methodologies, difficulties and solutions, with a special interest in model comparison, while the mathematics and economic intuitions of the model it evaluates are drawn from, and left to, \citet{CogleyNason1994}. If anything is left unanswered or vague in the description below, you may also want to refer to \citet{CogleyNason1994}.\\

In essence, the model studied by \citet{Schorfheide2000} is one of cash in advance (CIA). The goal of the paper is to estimate the model using Bayesian techniques, while observing only output and inflation. In the model, there are several markets and actors to keep track of. So to clarify things, figure \ref{fig:schorfmod} sketches the main dynamics of the model. As we go through the set-up of the model, you may find it helpful to refer back to the figure. 
\begin{figure} \label{fig:schorfmod}
\begin{center} 
\includegraphics[width=1.0\textwidth]{P_SchorfMod} 
\end{center} 
\caption[CIA model illustration]{Continuous lines show the circulation of nominal funds, while dashed lines show the flow of real variables.} 
\end{figure}\\

The economy is made up of three central agents and one secondary agent: households, firms and banks (representing the financial sector), and a monetary authority which plays a minor role. Households maximize their utility function which depends on consumption, $C_t$, and hours worked, $H_t$, while deciding how much money to hold next period in cash, $M_{t+1}$ and how much to deposit at the bank, $D_t$, in order to earn $R_{H,t}-1$ interest. Households therefore solve the problem
\begin{eqnarray*}
\substack{\max \\ \{C_t,H_t,M_{t+1},D_t\}} & \mathbb{E}_0 \left[ \sum_{t=0}^\infty \beta^t \left[ (1-\phi) \ln C_t + \phi \ln (1-H_t) \right] \right] \\
\textrm{s.t.} & P_t C_t \leq M_t - D_t + W_t H_t \\
& 0 \leq D_t \\
& M_{t+1} = (M_t - D_t + W_tH_t - P_tC_t) + R_{H,t}D_t + F_t + B_t
\end{eqnarray*}
where the second equation spells out the cash in advance constraint including wage revenues, the third the inability to borrow from the bank and the fourth the intertemporal budget constraint emphasizing that households accumulate the money that remains after bank deposits and purchases on goods are deducted from total inflows made up of the money they receive from last period's cash balances, wages, interests, as well as dividends from firms, $F_t$ and from banks, $B_t$ which in both cases are made up of net cash inflows defined below. \\

Banks, on their end, receive cash deposits from households and a cash injection, $X_t$ from the central bank (which equals the net change in nominal money balances, $M_{t+1}-M_t$). It uses these funds to disburse loans to firms, $L_t$, on which they make a net return of $R_{F,t}-1$. Of course, banks are constrained in their loans by a credit market equilibrium condition $L_t \leq X_t + D_t$. Finally, bank dividends, $B_t$ are simply equal to $D_t + R_{F,t}L_t - R_{H,t}D_t - L_t + X_t$. \\

Finally, firms maximize the net present value of future dividends (discounted by the marginal utility of consumption, since they are owned by households) by choosing dividends, next period's capital stock, $K_{t+1}$, labor demand, $N_t$ and loans. Its problem is summarized by
\begin{eqnarray*}
\substack{\max \\ \{F_t,K_{t+1},N_{t},L_t\}} & \mathbb{E}_0 \left[ \sum_{t=0}^\infty \beta^{t+1} \frac{F_t}{C_{t+1}P_{t+1}} \right] \\
\textrm{s.t.} & F_t \leq L_t + P_t \left[ K_t^\alpha (A_t N_t)^{1-\alpha} - K_{t+1} + (1-\delta)K_t \right] - W_tN_t-L_tR_{F,t} \\
& W_tN_t \leq L_t\\
\end{eqnarray*}
where the second equation makes use of the production function \mbox{$Y_t = K_t^\alpha (A_t N_t)^{1-\alpha}$} and the real aggregate accounting constraint (goods market equilibrium) \mbox{$C_t + I_t = Y_t$}, where $I_t=K_{t+1} - (1-\delta)K_t$, and where $\delta$ is the rate of depreciation. Note that it is the firms that engage in investment in this model, by trading off investment for dividends to consumers. The third equation simply specifies that bank loans are used to pay for wage costs. \\

To close the model, we add the usual labor and money market equilibrium equations, $H_t= N_t$ and $P_tC_t=M_t + X_t$, as well as the condition that $R_{H,t}=R_{F,t}$ due to the equal risk profiles of the loans.\\

More importantly, we add the stochastic elements to the model. The model allows for two sources of perturbations, one real affecting technology and one nominal affecting the money stock. These important equations are
\[
\ln A_t = \gamma + \ln A_{t-1} + \epsilon_{A,t}, \qquad \epsilon_{A,t} \thicksim N(0,\sigma_A^2)
\]
and

\[
\ln m_t = (1-\rho)\ln m^* + \rho \ln m_{t-1} + \epsilon_{M,t}, \qquad \epsilon_{M,t} \thicksim N(0,\sigma_M^2)
\]
where $m_t \equiv M_{T+1}/M_t$ is the growth rate of the money stock.\\

The first equation is therefore a unit root with drift in the log of technology, and the second an autoregressive stationary process in the growth rate of money, but an AR(2) in the level of money which has a unit root in the log of the level of money. (You can convince yourself of this by writing the AR(2) process in state space form and realizing that the system has an eigenvalue of one. Alternatively, one is a root of the second order autoregressive lag polynomial. As usual, if the logs of a variable are specified to follow a unit root process, the rate of growth of the series is a stationary stochastic process; see \citet{Hamilton1994}, chapter 15, for details.)\\

When the above functions are maximized, we obtain the following set of first order and equilibrium conditions. We will not dwell on the derivations here, to save space, but encourage you to browse \citet{CogleyNason1994} for additional details. We nonetheless give a brief intuitive explanation of each equation. The system comes down to
\begin{eqnarray*}
\mathbb{E}_{t}\bigg\{-\widehat{P}_{t}/\left[ \widehat{C}_{t+1}\widehat{P}_{t+1}m_{t}%
\right]\bigg\}&=&\beta e^{-\alpha (\gamma +\varepsilon_{t+1})}P_{t+1}\Big[\alpha\widehat{K}_{t}^{\alpha -1}N_{t+1}^{1-\alpha }+
(1-\delta )\Big]\\
&&/\left[ 
\widehat{c}_{t+2}\widehat{P}_{t+2}m_{t+1}\right] \bigg\}\\
\widehat{W}_{t}&=&\widehat{L}_{t}/N_{t}\\
\frac{\phi }{1-\phi }\left[ \widehat{C}_{t}\widehat{P}_{t}/\left(
1-N_{t}\right) \right]&=&\widehat{L}_{t}/N_{t}\\
R_{t}&=&(1-\alpha )\widehat{P}_{t}e^{-\alpha (\gamma +\varepsilon_{t+1})}\widehat{K}_{t-1}^{\alpha}N_{t}^{-\alpha }/\widehat{W}_{t}\\
\left[ \widehat{C}_{t}\widehat{P}_{t}\right] ^{-1}&=&\beta \left[ \left(
1-\alpha \right) \widehat{P}_{t}e^{-\alpha (\gamma +\varepsilon_{t+1})}\widehat{K}_{t-1}^{\alpha}N_{t}^{1-\alpha }
\right]\\
&& \times  \mathbb{E}_{t}\left[\widehat{L}_{t}m_{t}\widehat{C}_{t+1}\widehat{P}_{t+1}\right] ^{-1}\\
\widehat{C}_t+\widehat{K}_t &=& e^{-\alpha(\gamma+\varepsilon_t)}\widehat{K}_{t-1}^\alpha N^{1-\alpha}+(1-\delta)e^{-(\gamma+\varepsilon_t)}\widehat{K}_{t-1}\\
\widehat{P}_t\widehat{C} &=& m_t\\
m_t-1+\widehat{D}_t &=& \widehat{L}_t\\
\widehat{Y}_t &=& \widehat{K}_{t-1}^\alpha N^{1-alp}e^{-\alpha (\gamma+\varepsilon_t)}\\
\ln(m_t) &=& (1-\rho)\ln(m^\star) + \rho\ln(m_{t-1})+\eta_t\\
\frac{A_t}{A_{t-1}}  \equiv  dA_t & = & \exp( \gamma + \epsilon_{A,t})\\
Y_t/Y_{t-1} &=& e^{\gamma+\varepsilon_t}\widehat{Y}_t/\widehat{Y}_{t-1}\\
P_t/P_{t-1} &=& (\widehat{P}_t/\widehat{P}_{t-1})(m_{t-1}/e^{\gamma+\varepsilon_t})
\end{eqnarray*}
where, importantly, hats over variables no longer mean deviations from steady state, but instead represent variables that have been made stationary. We come back to this important topic in details in section \ref{sec:decmod} below on declaring the model in Dynare. For now, we pause a moment to give some intuition for the above equations. In order, these equations correspond to:
\begin{enumerate}
\item The Euler equation in the goods market, representing the tradeoff to the economy of moving consumption goods across time.
\item The firms' borrowing constraint, also affecting labor demand, as firms use borrowed funds to pay for labor input.
\item The intertemporal labor market optimality condition, linking labor supply, labor demand, and the marginal rate of substitution between consumption and leisure. 
\item The equilibrium interest rate in which the marginal revenue product of labor equals the cost of borrowing to pay for that additional unit of labor.
\item The Euler equation in the credit market, which ensures that giving up one unit of consumption today for additional savings equals the net present value of future consumption.
\item The aggregate resource constraint.
\item The money market equilibrium condition equating nominal consumption demand to money demand to money supply to current nominal balances plus money injection. 
\item The credit market equilibrium condition.
\item The production function.
\item The stochastic process for money growth.
\item The stochastic process for technology.
\item The relationship between observable variables and stationary variables; more details on these last two equations appear in the following section. 
\end{enumerate}

\section{Declaring variables}
To input the above model into Dynare for estimation purposes, we must first declare the mode's variables. This is done exactly as described in chapter \ref{ch:solbase} on solving DSGE models. We thus begin the .mod file with:\\
\\
\texttt{var m P c e W R k d n l gy\_obs gp\_obs Y\_obs P\_obs y dA; \\
varexo e\_a e\_m;}\\
\\
where the choice of upper and lower case letters is not significant, the first set of endogenous variables, up to $l$, are as specified in the model setup above, and where the last five variables are defined and explained in more details in the section below on declaring the model in Dynare. The exogenous variables are as expected and concern the shocks to the evolution of technology and  money balances. \\

\section{Declaring the model: from theory to Dynare} \label{sec:decmod}

The problem that must first be addressed is the non-stationarity of the variables originating from the stochastic trends in technology and money growth. In order to linearize and solve the DSGE model - a necessary first step before estimating it - we must have a steady state. Thus, we must write the model using stationary variables. Our task will then be to link these stationary variables to the data which, according to the basic (non-stationary) model, is non-stationary. Of course, we have the option to stationarize the data by looking at growth rates instead of levels, but can also work with non-stationary data in levels. The latter complicates things somewhat, but illustrates several features of Dynare worth highlighting; we therefore elect this path in our example below. But again, independently of the data, the model we input into Dynare must contain stationary variables, except for the observable variables. \\

Let us then first focus on this problem of non-stationarity. It is possible to show, by solving the above model, that it does not have a steady state. Instead, when shocks are null, real variables growth with $A_t$ (except for labor, $N_t$, which is stationary as there is no population growth), nominal variables grow with $M_t$ and prices with $M_t/A_t$. Stochastic detrending therefore involves the following operations (where hats over variables represent stationary variables). For real variables, $\hat q_t=q_t/A_t$, where $q_t = [y_t, c_t, i_t, k_{t+1} ]$. For nominal variables, $\hat{Q}_t = Q_t/M_t$, where $Q_t = [ d_t, l_t, W_t ]$. And for prices, $\hat P_t = P_t \cdot A_t / M_t$. \\

Let's illustrate this transformation on output, and leave the transformations of the remaining equations as an exercise (\citet{CogleyNason1994} includes more details on the transformations of each equation). We stationarize output by dividing its real variables (except for labor) by $A_t$. We define $\widehat{Y}_t$ to equal $Y_t / A_t$ and $\widehat{K}_t$ as $K_t / A_t$. \textsf{\textbf{TIP!}} Note, though, that in Dynare, the stock variables adopt the time subscript of the period in which they are decided (for more details, see section \ref{sec:modspe} in chapter \ref{ch:solbase}). Thus, in the output equation, we should actually work with $\widehat{K}_{t-1}=K_{t-1} / A_{t-1}$. The resulting equation made up of stationary variables is
\begin{eqnarray*}
\frac{Y_t}{A_t} & = & \left( \frac{K_{t-1}}{A_{t-1}} \right)^\alpha A_t^{1-\alpha} N_t^{1-\alpha} A_t^{-1} A_{t-1}^\alpha \\
\widehat{Y}_t & = & \widehat{K}_{t-1}^\alpha N_t^{1-\alpha} \left( \frac{A_t}{A_{t-1}} \right)^{-\alpha} \\
& = & \widehat{K}_{t-1}^\alpha N_t^{1-\alpha} \exp(-\alpha (\gamma + \epsilon_{A,t}))
\end{eqnarray*}
where we go from the second to the third line by taking the exponential of both sides of the equation of motion of technology.\\

The above is the equation we retain for the .mod file of Dynare into which we type:\\
\\
\texttt{y=k(-1)$\hat{}$alp*n$\hat{}$(1-alp)*exp(-alp*(gam+e\_a))}\\
\\

The other equations are entered into the .mod file after transforming them in exactly the same way as the one above. But before presenting all final equations, let's consider two additional transformations. First, since we often deal with the growth rate of technology, we specify the equation of motion of technology as\\
\\
\texttt{dA = exp(gam+e\_a)}\\
\\
by simply taking the exponential of both sides of the stochastic process of technology defined in the model setup above. \\

And finally, we must make a decision as to what we do with our non-stationary observations. We could simply stationarize them by looking at the rates of growth (which we know are constant). In the case of output, the observable variable would become $Y_t /Y_{t-1}$. We would then have to relate this to our stationary variables by using the definition that $\widehat{Y}_t \equiv Y_t/ A_t$:\\
\\
\texttt{gy\_obs = dA*y/y(-1);}\\
\\

But, as we suggested earlier, since we want to work with the non-stationary series directly, we separate out $Y_t$ and $Y_{t-1}$ in the above, and making use of the identity $Y_t/Y_{t-1}=(\widehat{Y}_t/\widehat{Y}_{t-1})\cdot \exp (\gamma + \epsilon_{A,t})$, we add to the .mod file\\
\\
\texttt{Y\_obs/Y\_obs(-1) = gy\_obs}\\
\\
We of course do the same for prices, our other observable variable, except that prices grow with $M_t/A_t$ as noted earlier. The details of the correct transformations for prices are left as an exercise.\\

The resulting model block of the .mod file is therefore:\\
\\
\texttt{model;\\
dA = exp(gam+e\_a);\\
log(m) = (1-rho)*log(mst) + rho*log(m(-1))+e\_m;\\
-P/(c(+1)*P(+1)*m)+bet*P(+1)*(alp*exp(-alp*(gam+log(e(+1))))*k$\hat{}$(alp-1)\\
*n(+1)$\hat{}$(1-alp)+(1-del)*exp(-(gam+log(e(+1)))))/(c(+2)*P(+2)*m(+1))=0;\\
W = l/n;\\
-(psi/(1-psi))*(c*P/(1-n))+l/n = 0;\\
R = P*(1-alp)*exp(-alp*(gam+e\_a))*k(-1)$\hat{}$alp*n$\hat{}$(-alp)/W;\\
1/(c*P)-bet*P*(1-alp)*exp(-alp*(gam+e\_a))*k(-1)$\hat{}$alp*n$\hat{}$(1-alp)/(m*l*c(+1)*P(+1)) = 0;\\
c+k = exp(-alp*(gam+e\_a))*k(-1)$\hat{}$alp*n$\hat{}$(1-alp)+(1-del)*exp(-(gam+e\_a))*k(-1);\\
P*c = m;\\
m-1+d = l;\\
e = exp(e\_a);\\
y = k(-1)$\hat{}$alp*n$\hat{}$(1-alp)*exp(-alp*(gam+e\_a));\\
gy\_obs = dA*y/y(-1);\\
gp\_obs = (p/p(-1))*m(-1)/dA;\\
Y\_obs/Y\_obs(-1) = gy\_obs;\\
P\_obs/P\_obs(-1) = gp\_obs;\\
end;}\\
\\
where, of course, the input conventions, such as ending lines with semicolons and indicating the timing of variables in parentheses, are the same as those listed in chapter \ref{ch:solbase}. \\

\section{Specifying variable characteristics}
A natural extension of our discussion above is to declare, in Dynare, which variables are observable and which are non-stationary. We begin with the first and declare, in our .mod file\\
\\
\texttt{varobs P\_obs Y\_obs;}\\
\\
to specify that our observable variables are indeed P\_obs and Y\_obs as noted in the section above. \textsf{\textbf{TIP!}} Note that the number of unobserved variables (number of \texttt{var} minus number of \texttt{varobs}) must be smaller or equal to the number of shocks such that the model be estimated. If this is not the case, you should add measurement shocks to your model where you deem most appropriate. \\

Secondly, we also recall that we decided to work with the levels of these variables, which are not stationary. In fact, both exhibit stochastic trends. This can be seen explicitly by taking the difference of logs of output and prices to compute growth rates. In the case of output, we make use of the fact that $Y_t=\widehat Y_t \cdot A_t$. Taking logs of both sides and subtracting the same equation scrolled back one period, we find:
\[
\Delta \ln Y_t = \Delta \ln \widehat Y_t + \gamma + \epsilon_{A,t}
\]
emphasizing clearly the drift term $\gamma$, whereas we know $\Delta \ln \widehat Y_t$ is stationary in steady state. \\

In the case of prices, we apply the same manipulations to show that:
\[
\Delta \ln P_t = \Delta \ln \widehat P_t + \ln m_{t-1} - \gamma - \epsilon_{A,t}
\]
again, highlighting the trend terms clearly. \\

In Dynare, these trends must be declared up front. We therefore write\\
\\
\texttt{observation\_trends;\\
P\_obs (log(exp(gam)/mst));\\
Y\_obs (gam);\\
end;}\\
(** not sure I get the trend declaration for prices... how does it match the growth rate computed above? mst is simply the steady state growth of m?). In general, the command \texttt{observation\_trends} specifies linear trends as a function of model parameters for the observed variables in the model.\\

And finally, since P\_obs and Y\_obs inherit the unit root characteristics of their driving variables, technology and money, we must tell Dynare to use a diffuse prior (infinite variance) for their initialization in the Kalman filter. Note that for stationary variables, the unconditional covariance matrix of these variables is used for initialization. The algorithm to compute a true diffuse prior is taken from \citet{DurbinKoopman2001}. To give these instructions to Dynare, we write in the .mod\\
\\
\texttt{options\_.unit\_root\_vars = \{'P\_obs'; 'Y\_obs' \};}\\
\\
(** is this only if we have a true unit root, or also if we have non-stationarity coming from a deterministic trend, for instance?)\\

\section{Specifying the steady state}
Before Dynare estimates a model, it first linearizes it around a steady state. Thus, a steady state must exist for the model and although Dynare can calculate it, we must give it a hand by declaring approximate values for the steady states. This is just as explained in details and according to the same syntax explained in chapter \ref{ch:solbase}, with the convenient options to use the commands \texttt{steady} and \texttt{check}. The only difference with regards to model estimation is the non-stationary nature of our observable variables. In this case, the declaration of the steady state must be slightly amended in Dynare. (** fill in with the new functionality from Dynare version 4, currently the answer is to write a separate Matlab file which solves the steady state for the stationary variables and returns dummies for the non stationary variables). 

\section{Declaring priors}
Priors play an important role in Bayesian estimation and consequently deserve a central role in the specification of the .mod file. Priors, in Bayesian estimation, are declared as a distribution. The general syntax to introduce priors in Dynare is the following:\\
\\
\texttt{estimated\_params;\\
PARAMETER NAME, PRIOR\_SHAPE, PRIOR\_MEAN, PRIOR\_STANDARD\_ERROR [, PRIOR 3$^{\textrm{rd}}$ PARAMETER] [,PRIOR 4$^{\textrm{th}}$ PARAMETER] ; \\
end; }\\
\\
where the following table defines each term more clearly\\
\\
    \begin{tabular}{llc}
    \textbf{\underline{PRIOR\_SHAPE}} & \textbf{\underline{Corresponding distribution}} & \textbf{\underline{Range}} \\
      NORMAL\_PDF & $N(\mu,\sigma)$ & $\mathbb{R}$\\
      GAMMA\_PDF &  $G_2(\mu,\sigma,p_3)$ & $[p_3,+\infty)$\\
        BETA\_PDF & $B(\mu,\sigma,p_3,p_4)$ & $[p_3,p_4]$\\
        INV\_GAMMA\_PDF & $IG_1(\mu,\sigma)$ & $\mathbb R^+$\\
UNIFORM\_PDF& $U(p_3,p_4)$ &  $[p_3,p_4]$
    \end{tabular}\\
    \\
    \\
    where $\mu$ is the \texttt{PRIOR\_MEAN}, $\sigma$ is the \texttt{PRIOR\_STANDARD\_ERROR}, $p_3$ is the \texttt{PRIOR 3$^{\textrm{rd}}$ PARAMETER} (whose default is 0) and $p_4$ is the \texttt{PRIOR 4$^{\textrm{th}}$ PARAMETER} (whose default is 1). \\
    
For a more complete review of all possible options for declaring priors, as well as the syntax to declare priors for maximum likelihood estimation (not Bayesian), see the Reference Manual (** add link). Note also that if some parameters in a model are calibrated and are not to be estimated, you should declare them as such, by using the \texttt{parameters} command and its related syntax, as explained in chapter \ref{ch:solbase}. \\
    
\textsf{\textbf{TIP!}} It may at times be desirable to estimate a transformation of a parameter appearing in the model, rather than the parameter itself. In such a case, it is possible to declare the parameter to be estimated in the parameters statement and to define the transformation at the top of the model section, as a Matlab expression, by adding a pound sign (\#) at the beginning of the corresponding line. For example:\\
\\
\texttt{model;\\
      \# sig = 1/bet;\\
      c = sig*c(+1)*mpk;\\
      end;\\
\\
      estimated\_params;\\
      bet,normal\_pdf,1,0.05;\\
      end;}\\

\textsf{\textbf{TIP!}} Finally, another useful command to use is the \texttt{estimated\_params\_init} command which declares numerical initial values for the optimizer when these one are different from the prior mean. This is especially useful when redoing an estimation - if the optimizer got stuck the first time around, or if the number of Metropolis-Hastings draws was insufficient - and wanting to enter the posterior mode as initial values for the parameters instead of a prior. The Reference Manual (** add link) gives more details as to the exact syntax of this command.\\

Coming back to our example and having learned how to enter priors, we expand our .mod file with the following information: \\
\\
\texttt{estimated\_params;\\
alp, beta\_pdf, 0.356, 0.02; \\ 
bet, beta\_pdf, 0.993, 0.002; \\
gam, normal\_pdf, 0.0085, 0.003; \\
mst, normal\_pdf, 1.0002, 0.007; \\
rho, beta\_pdf, 0.129, 0.223;\\
psi, beta\_pdf, 0.65, 0.05;\\
del, beta\_pdf, 0.01, 0.005;\\
stderr e\_a, inv\_gamma\_pdf, 0.035449, inf;\\
stderr e\_m, inv\_gamma\_pdf, 0.008862, inf;\\
end;}\\
\\
(** do you still need to declare all parameters using the ``parameter'' command at the top of the .mod file?)\\

\section{Initiating the estimation} \label{sec:estimate}
To ask Dynare to estimate a model, all that is necessary is to add the command \texttt{estimation} at the end of the .mod file. Easy enough. But the real complexity comes from the options available for the command (to be entered in parentheses and sequentially, separated by commas, after the command \texttt{estimation}). Below, we list the most common and useful options, and encourage you to view the Reference Manual (** add link) for a complete list. \\
\begin{enumerate}
\item datafile = FILENAME: the datafile (a .m file, a .mat file, or an .xls file). Note that observations do not need to show up in any order, but vectors of observations need to be named with the same names as those in \texttt{var\_obs}. In Excel files, for instance, observations could be ordered in columns, and variable names would show up in the first cell of each column.
\item nobs = INTEGER: the number of observations to be used (default: all observations in the file) 
\item first\_obs = INTEGER: the number of the first observation to be used (default = 1). This is useful when running loops, or instance, to divide the observations into sub-periods. 
\item prefilter = 1: the estimation procedure demeans the data (default=0, no prefiltering). This is useful if model variables are in deviations from steady state, for instance, and therefore have zero mean. Demeaning the observations would also impose a zero mean on the observed variables.  
\item nograph: no graphs should be plotted.
\item lik\_init: INTEGER: type of initialization of Kalman filter. 
\subitem 1 (default): for stationary models, the initial matrix of variance of the error of forecast is set equal to the 
unconditional variance of the state variables. 
\subitem 2: for nonstationary models: a wide prior is used with an initial matrix of variance of the error of forecast 
diagonal with 10 on the diagonal. When using \texttt{unit\_root\_vars}, the \texttt{like\_init} option as no effect. (** but if have both stationary and nonstationary variables, what option should you use?)
\item conf\_sig = \{INTEGER | DOUBLE \}: the level for the confidence intervals reported in the results (default = 0.90) 
\item mh\_replic = INTEGER: number of replication for Metropolis Hasting algorithm. For the time being, mh\_replic 
should be larger than 1200 (default = 20000) ** is this still true? 
\item mh\_nblocks = INTEGER: number of parallel chains for Metropolis Hasting algorithm (default = 2). Despite this low default value, it is advisable to work with a higher value, such as 5 or more. This improves the computation of between group variance of the parameter means, one of the key criteria to evaluate the efficiency of the Metropolis-Hastings to evaluate the posterior distribution. More details on this subject appear in Chapter \ref{ch:estadv}.  
\item mh\_drop = DOUBLE: the fraction of initially generated parameter vectors to be dropped before using posterior 
simulations (default = 0.5) ** what does initially generated mean?
\item mh\_jscale = DOUBLE: the scale to be used for the jumping distribution in MH algorithm. The default value is 
rarely satisfactory. This option must be tuned to obtain, ideally, an accpetation rate of 25\% in the Metropolis- Hastings algorithm (default = 0.2). The idea is not to reject or accept too often a candidate parameter; the literature has settled on a value of approximately 25\%. If the acceptance rate were too high, your Metropolis-Hastings iterations would never visit the endzones of a distribution, while if it were too low, the iterations would get stuck in a subspace of the parameter range. Note that the acceptance rate drops if you increase the scale used in the jumping distribution and vice a versa.
\item mh\_init\_scale=DOUBLE: the scale to be used for drawing the initial value of the Metropolis-Hastings chain (default=2*mh\_jscale). The idea here is to draw initial values from a stretched out distribution in order to maximize the chances of these values not being too close together, which would defeat the purpose of running several blocks of Metropolis-Hastings chains.
\item mode\_file=FILENAME: name of the file containing previous value for the mode. When computing the mode, 
Dynare stores the mode (xparam1) and the hessian (hh) in a file called MODEL NAME\_mode. This is a particularly helpful option to speed up the estimation process if you have already undertaken initial estimations and have values of the posterior mode. 
\item mode\_compute=INTEGER: specifies the optimizer for the mode computation. 
\subitem 0: the mode isn�t computed. mode\_file must be specified 
\subitem 1: uses Matlab fmincon (see the Reference Manual (** add link) to set options for this command).
\subitem 2: uses Lester Ingber�s Adaptive Simulated Annealing. 
\subitem 3: uses Matlab fminunc. 
\subitem 4 (default): uses Chris Sim�s csminwel. 
\item mode\_check: when mode\_check is set, Dynare plots the posterior density for values around the computed mode 
for each estimated parameter in turn. This is helpful to diagnose problems with the optimizer. A clear indication of a problem would be that the mode is not at the trough of a distribution. 
\item load\_mh\_file: when load\_mh\_file is declared, Dynare adds to previous Metropolis-Hastings simulations instead 
of starting from scratch. Again, this is a useful option to speed up the process of estimation. 
\item nodiagnostic: doesn�t compute the convergence diagnostics for Metropolis-Hastings (default: diagnostics are computed and 
displayed). Actually seeing if the various blocks of Metropolis-Hastings runs converge is a powerful and useful option to build confidence in your model estimation. More details on these diagnostics are given in Chapter \ref{ch:estadv}.
\item bayesian\_irf: triggers the computation of the posterior distribution of impulse response functions (IRFs). The length of the IRFs are controlled by the irf option (** where, how?). The shocks at the origin of these IRFs are those estimated by Dynare. 
\item moments\_varendo: triggers the computation of the posterior distribution of the theoretical moments of the endogenous variables as in \texttt{stoch\_simul} (the posterior distribution of the variance decomposition is also included). 
\item filtered\_vars: triggers the computation of the posterior distribution of filtered endogenous variables and shocks. See the note below on the difference between filtered and smoothed shocks. 
\item smoother: triggers the computation of the posterior distribution of smoothed endogenous variables and shocks. Smoothed shocks are a reconstruction of the values of unobserved shocks over the sample, using all the information contained in the sample of observations. Filtered shocks, instead, are built only based on knowing past information. To calculate one period ahead prediction errors, for instance, you should use filtered, not smoothed variables.
\item forecast = INTEGER: computes the posterior distribution of a forecast on INTEGER periods after the end of the sample used in estimation. The corresponding graph includes one confidence interval describing uncertainty due to parameters and one confidence interval describing uncertainty due to parameters and future shocks. Note that Dynare cannot forecast out of the posterior mode. You need to run Metropolis-Hastings iterations before being able to run forecasts on an estimated model.
\item tex requests the printing of results and graphs in TeX tables and graphics that can be later directly included in Latex files (** any doc on this?)
\item All options available for stoch\_simul can simply be added to the above options, separated by commas. To view a list of these options, either see the Reference Manual (** add link) or section \ref{sec:compute} of Chapter \ref{ch:solbase}. (** how do you run stoch\_simul after estimation so that the shocks to be used in the simulation are those estimated and parameters are from the posterior mode?)
\end{enumerate}

\textsf{\textbf{TIP!}} Before launching estimation it is a good idea to make sure that your model is correctly declared, that a steady state exists and that it can be simulated for at least one set of parameter values. You may therefore want to create a test version of your .mod file. In this test file, you would comment out or erase the commands related to estimation, remove the non-stationary variables from your model and instead add a \texttt{shocks} block, make sure you have \texttt{steady} and \texttt{check} following the \texttt{initval} block if you do not have exact steady state values and run a simulation using \texttt{stoch\_simul} at the end of your .mod file. Details on model solution and simulation can be found in Chapter \ref{ch:solbase}. \\

Finally, coming back to our example, we would add the following commands to ask Dynare to run a basic estimation of our model:\\
\\
\texttt{estimation(datafile=fsdat,nobs=192,loglinear,mh\_replic=5000,\\
mh\_nblocks=5,mh\_drop=0.45,mh\_jscale=0.8);}\\

This ends our description of the .mod file.

\section{The complete .mod file}

We have seen each part of the .mod separately; it's now time to get a picture of what the complete file looks like. For convenience, the file also appears in the ``examples'' folder of your Dynare installation, under \texttt{fs2000l.mod}. (** rewrite and replace example file. Also, make sure the .mod below works given the slightly rearranged order of each block. Finally, change the initval block below with the appropriate commands for Dynare version 4, with respect to the non stationary variables). \\
\\
\texttt{var m P c e W R k d n l gy\_obs gp\_obs Y\_obs P\_obs y dA; \\
varexo e\_a e\_m;\\
\\
model;\\
dA = exp(gam+e\_a);\\
log(m) = (1-rho)*log(mst) + rho*log(m(-1))+e\_m;\\
-P/(c(+1)*P(+1)*m)+bet*P(+1)*(alp*exp(-alp*(gam+log(e(+1))))*k$\hat{}$(alp-1)\\
*n(+1)$\hat{}$(1-alp)+(1-del)*exp(-(gam+log(e(+1)))))/(c(+2)*P(+2)*m(+1))=0;\\
W = l/n;\\
-(psi/(1-psi))*(c*P/(1-n))+l/n = 0;\\
R = P*(1-alp)*exp(-alp*(gam+e\_a))*k(-1)$\hat{}$alp*n$\hat{}$(-alp)/W;\\
1/(c*P)-bet*P*(1-alp)*exp(-alp*(gam+e\_a))*k(-1)$\hat{}$alp*n$\hat{}$(1-alp)/(m*l*c(+1)*P(+1)) = 0;\\
c+k = exp(-alp*(gam+e\_a))*k(-1)$\hat{}$alp*n$\hat{}$(1-alp)+(1-del)*exp(-(gam+e\_a))*k(-1);\\
P*c = m;\\
m-1+d = l;\\
e = exp(e\_a);\\
y = k(-1)$\hat{}$alp*n$\hat{}$(1-alp)*exp(-alp*(gam+e\_a));\\
gy\_obs = dA*y/y(-1);\\
gp\_obs = (p/p(-1))*m(-1)/dA;\\
Y\_obs/Y\_obs(-1) = gy\_obs;\\
P\_obs/P\_obs(-1) = gp\_obs;\\
end;\\
\\
varobs P\_obs Y\_obs;\\
\\
observation\_trends;\\
P\_obs (log(exp(gam)/mst));\\
Y\_obs (gam);\\
end;\\
\\
options\_unit\_root\_vars = \{'P\_obs'; 'Y\_obs' \};\\
\\
initval;\\
k = 6;\\
m = mst;\\
P = 2.25;\\
c = 0.45;\\
e = 1;\\
W = 4;\\
R = 1.02;\\
d = 0.85;\\
n = 0.19;\\
l = 0.86;\\
y = 0.6;\\
gy\_obs = exp(gam);\\
gp\_obs = exp(-gam); \\
dA = exp(gam);\\
end;\\
\\
steady;\\
\\
check;\\
\\
estimated\_params;\\
alp, beta\_pdf, 0.356, 0.02; \\ 
bet, beta\_pdf, 0.993, 0.002; \\
gam, normal\_pdf, 0.0085, 0.003; \\
mst, normal\_pdf, 1.0002, 0.007; \\
rho, beta\_pdf, 0.129, 0.223;\\
psi, beta\_pdf, 0.65, 0.05;\\
del, beta\_pdf, 0.01, 0.005;\\
stderr e\_a, inv\_gamma\_pdf, 0.035449, inf;\\
stderr e\_m, inv\_gamma\_pdf, 0.008862, inf;\\
end;\\
\\
estimation(datafile=fsdat,nobs=192,loglinear,mh\_replic=5000,\\
mh\_nblocks=5,mh\_drop=0.45,mh\_jscale=0.8);}\\
\\

\section{Summing it up}
The explanations given above of each step necessary to translate the \citet{Schorfheide2000} example into language that Dynare can understand and process was quite lengthy and involved a slew of new commands and information. It may therefore be useful, to gain a ``bird's eyeview'' on what we have just accomplished, and summarize the most important steps at a high level. This is done in figure \ref{fig:estsumm}.\\
\begin{figure} \label{fig:estsumm}
\begin{center} 
\includegraphics[width=1.0\textwidth]{P_flowest} 
\end{center} 
\caption[Steps of model estimation]{At a high level, there are five basic steps to translate a model into Dynare for successful estimation.} 
\end{figure}\\

\section{Interpreting output}

As in the case of model solution and simulation, Dynare returns both tabular and graphical output. On the basis of the options entered in the example .mod file above, Dynare will display the following results.\\

\subsection{Tabular results}
The first results to be displayed (and calculated from a chronological standpoint) are the steady state results. Note the dummy values of 1 for the non-stationary variables Y\_obs and P\_obs. These results are followed by the eigenvalues of the system, presented in the order in which the endogenous variables are declared at the beginning of the .mod file (** is this true?). The table of eigenvalues is completed with a statement about the Blanchard-Kahn condition being met - hopefully!\\

The next set of results are for the numerical iterations necessary to find the posterior mode, as explained in more details in Chapter \ref{ch:estadv}. The improvement from one iteration to the next reaches zero, Dynare give the value of the objective function (the posterior Kernel) at the mode and displays two important table summarizing results from posterior maximization.\\

The first table summarizes results for parameter values. It includes: prior means, posterior mode, standard deviation and t-stat of the mode (based on the assumption of a Normal, probably erroneous when undertaking Bayesian estimation, as opposed to standard maximum likelihood), as well as the prior distribution and standard deviation (pstdev). It is followed by a second table summarizing the same results for the shocks. (** what does it mean when shocks have pstdev Inf?)\\

\subsection{Graphical results}
(** intersperse screen shots below as each output is discussed)\\

The first figure comes up soon after launching Dynare as little computation is necessary to generate it. The figure returns a graphical representation of the priors for each parameter of interest. \\

The second set of figures displays ``MCMC univariate diagnostics'', where MCMC stands for Monte Carlo Markov Chains. This is the main source of feedback to gain confidence, or spot a problem, with results. Recall that Dynare completes several runs of Metropolis-Hastings simulations (as many as determined in the option \texttt{mh\_nblocks}, each time starting from a different initial value). If the results from one chain are sensible, and the optimizer did not get stuck in an odd area of the parameter subspace, two things should happen. First, results within any of the however many iterations of Metropolis-Hastings simulation should be similar. And second, results between the various chains should be close. This is the idea of what the MCMC diagnostics track. \\

More specifically, the red and blue lines on the charts represent specific measures of the parameter vectors both within and between chains. For the results to be sensible, these should be relatively constant (although there will always be some variation) and they should converge. Dynare reports three measures: ``interval'', being constructed from an 80\% confidence interval around the parameter mean, ``m2'', being a measure of the variance and ``m3'' based on third moments. In each case, Dynare reports both the within and the between chains measures. The figure entitled ``multivariate diagnostic'' presents results of the same nature, except that they reflect an aggregate measure based on the eigenvalues of the variance-covariance matrix of each parameter.\\

In our example above, you can tell that indeed, we obtain convergence and relative stability in all measures of the parameter moments. Note that the horizontal axis represents the number of Metropolis-Hastings iterations that have been undertaken, and the vertical axis the measure of the parameter moments, with the first, corresponding to the measure at the initial value of the Metropolis-Hastings iterations.\\

\textsf{\textbf{TIP!}} If the plotted moments are highly unstable or do not converge, you may have a problem of poor priors. It is advisable to redo the estimation with different priors. If you have trouble coming up with a new prior, try starting with a uniform and relatively wide prior and see where the data leads the posterior distribution. Another approach is to undertake a greater number of Metropolis-Hastings simulations.\\

The first to last figure - figure 6 in our example - displays the most interesting set of results, towards which most of the computations undertaken by Dynare are directed: the posterior distribution. In fact, the figure compares the posterior to the prior distribution (black vs. grey lines). In addition, on the posterior distribution, Dynare plots a green line which represents the posterior mode.\\

\textsf{\textbf{TIP!}} These graphs are not highly relevant and interesting for the results they display, but also as tools to detect problems or build additional confidence in your results. First, the prior and the posterior distributions should not be excessively different. Second, the posterior distributions should be close to normal, or at least not display a shape that is clearly non-normal. Third, the green mode (calculated from the numerical optimization of the posterior kernel) should not be too far away from the mode of the posterior distribution. If not, it is advisable to undertake a greater number of Metropolis-Hastings simulations. \\

The last figure returns the smoothed estimated shocks in a useful illustration to eye-ball the plausibility of the size and frequency of the shocks. The horizontal axis, in this case, represents the number of periods in the sample. One thing to check is the fact that shocks should be centered around zero. That is indeed the case for our example. \\

 







\chapter{Estimating DSGE models - advanced topics} \label{ch:estadv}

This chapter focusses on advanced topics and features of Dynare in the area of model estimation. The chapter begins by presenting a more complex example than the one used for illustration purposes in chapter \ref{ch:estbase}. The goal is to show how Dynare would be used in the more ``realistic'' setting of reproducing a recent academic paper. The chapter then follows with sections on comparing models to one another, and then to BVARs, and ends with a table summarizing where output series are stored and how these can be retrieved. \\


\section{Alternative and non-stationary example}
The example provided in chapter \ref{ch:estbase} is really only useful for illustration purposes. So we thought you would enjoy (and continue learning from!) a more realistic example which reproduces the work in a recent - and highly regarded - academic paper. The example shows how to use Dynare in a more realistic setting, while emphasizing techniques to deal with non-stationary observations and stochastic trends in dynamics. \\

\subsection{Introducing the example}
The example is drawn from \citet{Schorfheide2000}. This first section introduces the model, its basic intuitions and equations. We will then see in subsequent sections how to estimate it using Dynare. Note that the original paper by Schorfheide mainly focusses on estimation methodologies, difficulties and solutions, with a special interest in model comparison, while the mathematics and economic intuitions of the model it evaluates are drawn from \citet{CogleyNason1994}. That paper should serve as a helpful reference if anything is left unclear in the description below.\\

In essence, the model studied by \citet{Schorfheide2000} is one of cash in advance (CIA). The goal of the paper is to estimate the model using Bayesian techniques, while observing only output and inflation. In the model, there are several markets and actors to keep track of. So to clarify things, figure \ref{fig:schorfmod} sketches the main dynamics of the model. You may want to refer back to the figure as you read through the following sections. 
\begin{figure} \label{fig:schorfmod}
\begin{center} 
\includegraphics[width=1.0\textwidth]{P_SchorfMod} 
\end{center} 
\caption[CIA model illustration]{Continuous lines show the circulation of nominal funds, while dashed lines show the flow of real variables.} 
\end{figure}\\

The economy is made up of three central agents and one secondary agent: households, firms and banks (representing the financial sector), and a monetary authority which plays a minor role. Households maximize their utility function which depends on consumption, $C_t$, and hours worked, $H_t$, while deciding how much money to hold next period in cash, $M_{t+1}$ and how much to deposit at the bank, $D_t$, in order to earn $R_{H,t}-1$ interest. Households therefore solve the problem
\begin{eqnarray*}
\substack{\max \\ \{C_t,H_t,M_{t+1},D_t\}} & \mathbb{E}_0 \left[ \sum_{t=0}^\infty \beta^t \left[ (1-\phi) \ln C_t + \phi \ln (1-H_t) \right] \right] \\
\textrm{s.t.} & P_t C_t \leq M_t - D_t + W_t H_t \\
& 0 \leq D_t \\
& M_{t+1} = (M_t - D_t + W_tH_t - P_tC_t) + R_{H,t}D_t + F_t + B_t
\end{eqnarray*}
where the second equation spells out the cash in advance constraint including wage revenues, the third the inability to borrow from the bank and the fourth the intertemporal budget constraint emphasizing that households accumulate the money that remains after bank deposits and purchases on goods are deducted from total inflows made up of the money they receive from last period's cash balances, wages, interests, as well as dividends from firms, $F_t$, and from banks, $B_t$, which in both cases are made up of net cash inflows defined below. \\

Banks, on their end, receive cash deposits from households and a cash injection, $X_t$ from the central bank (which equals the net change in nominal money balances, $M_{t+1}-M_t$). It uses these funds to disburse loans to firms, $L_t$, on which they make a net return of $R_{F,t}-1$. Of course, banks are constrained in their loans by a credit market equilibrium condition $L_t \leq X_t + D_t$. Finally, bank dividends, $B_t$ are simply equal to $D_t + R_{F,t}L_t - R_{H,t}D_t - L_t + X_t$. \\

Finally, firms maximize the net present value of future dividends (discounted by the marginal utility of consumption, since they are owned by households) by choosing dividends, next period's capital stock, $K_{t+1}$, labor demand, $N_t$, and loans. Its problem is summarized by
\begin{eqnarray*}
\substack{\max \\ \{F_t,K_{t+1},N_{t},L_t\}} & \mathbb{E}_0 \left[ \sum_{t=0}^\infty \beta^{t+1} \frac{F_t}{C_{t+1}P_{t+1}} \right] \\
\textrm{s.t.} & F_t \leq L_t + P_t \left[ K_t^\alpha (A_t N_t)^{1-\alpha} - K_{t+1} + (1-\delta)K_t \right] - W_tN_t-L_tR_{F,t} \\
& W_tN_t \leq L_t\\
\end{eqnarray*}
where the second equation makes use of the production function \mbox{$Y_t = K_t^\alpha (A_t N_t)^{1-\alpha}$} and the real aggregate accounting constraint (goods market equilibrium) \mbox{$C_t + I_t = Y_t$}, where $I_t=K_{t+1} - (1-\delta)K_t$, and where $\delta$ is the rate of depreciation. Note that it is the firms that engage in investment in this model, by trading off investment for dividends to consumers. The third equation simply specifies that bank loans are used to pay for wage costs. \\

To close the model, we add the usual labor and money market equilibrium equations, $H_t= N_t$ and $P_tC_t=M_t + X_t$, as well as the condition that $R_{H,t}=R_{F,t}$ due to the equal risk profiles of the loans.\\

More importantly, we add a stochastic elements to the model. The model allows for two sources of perturbations, one real, affecting technology and one nominal, affecting the money stock. These important equations are
\[
\ln A_t = \gamma + \ln A_{t-1} + \epsilon_{A,t}, \qquad \epsilon_{A,t} \thicksim N(0,\sigma_A^2)
\]
and

\[
\ln m_t = (1-\rho)\ln m^* + \rho \ln m_{t-1} + \epsilon_{M,t}, \qquad \epsilon_{M,t} \thicksim N(0,\sigma_M^2)
\]
where $m_t \equiv M_{T+1}/M_t$ is the growth rate of the money stock. Note that theses expressions for trends are not written in the most straightforward manner nor very consistently. But we reproduced them never-the-less to make it easier to compare this example to the original paper. \\

The first equation is therefore a unit root with drift in the log of technology, and the second an autoregressive stationary process in the growth rate of money, but an AR(2) with a unit root in the log of the level of money. This can be seen from the definition of $m_t$ which can be rewritten as $\ln M_{t+1} = \ln M_t + \ln m_t$.\footnote{Alternatively, we could have written the AR(2) process in state space form and realized that the system has an eigenvalue of one. Otherwise said, one is a root of the second order autoregressive lag polynomial. As usual, if the logs of a variable are specified to follow a unit root process, the rate of growth of the series is a stationary stochastic process; see \citet{Hamilton1994}, chapter 15, for details.}\\

When the above functions are maximized, we obtain the following set of first order and equilibrium conditions. We will not dwell on the derivations here, to save space, but encourage you to browse \citet{CogleyNason1994} for additional details. We nonetheless give a brief intuitive explanation of each equation. The system comes down to
\begin{eqnarray*}
\mathbb{E}_{t}\bigg\{-\widehat{P}_{t}/\left[ \widehat{C}_{t+1}\widehat{P}_{t+1}m_{t}%
\right]\bigg\}&=&\beta e^{-\alpha (\gamma +\epsilon_{A,t+1})}P_{t+1}\Big[\alpha\widehat{K}_{t}^{\alpha -1}N_{t+1}^{1-\alpha }+
(1-\delta )\Big]\\
&&/\left[ 
\widehat{c}_{t+2}\widehat{P}_{t+2}m_{t+1}\right] \bigg\}\\
\widehat{W}_{t}&=&\widehat{L}_{t}/N_{t}\\
\frac{\phi }{1-\phi }\left[ \widehat{C}_{t}\widehat{P}_{t}/\left(
1-N_{t}\right) \right]&=&\widehat{L}_{t}/N_{t}\\
R_{t}&=&(1-\alpha )\widehat{P}_{t}e^{-\alpha (\gamma +\epsilon_{A,t+1})}\widehat{K}_{t-1}^{\alpha}N_{t}^{-\alpha }/\widehat{W}_{t}\\
\left[ \widehat{C}_{t}\widehat{P}_{t}\right] ^{-1}&=&\beta \left[ \left(
1-\alpha \right) \widehat{P}_{t}e^{-\alpha (\gamma +\epsilon_{A,t+1})}\widehat{K}_{t-1}^{\alpha}N_{t}^{1-\alpha }
\right]\\
&& \times  \mathbb{E}_{t}\left[\widehat{L}_{t}m_{t}\widehat{C}_{t+1}\widehat{P}_{t+1}\right] ^{-1}\\
\widehat{C}_t+\widehat{K}_t &=& e^{-\alpha(\gamma+\epsilon_{A,t})}\widehat{K}_{t-1}^\alpha N^{1-\alpha}+(1-\delta)e^{-(\gamma+\epsilon_{A,t})}\widehat{K}_{t-1}\\
\widehat{P}_t\widehat{C} &=& m_t\\
m_t-1+\widehat{D}_t &=& \widehat{L}_t\\
\widehat{Y}_t &=& \widehat{K}_{t-1}^\alpha N^{1-\alpha}e^{-\alpha (\gamma+\epsilon_{A,t})}\\
\ln(m_t) &=& (1-\rho)\ln(m^\star) + \rho\ln(m_{t-1})+\epsilon_{M,t}\\
\frac{A_t}{A_{t-1}}  \equiv  dA_t & = & \exp( \gamma + \epsilon_{A,t})\\
Y_t/Y_{t-1} &=& e^{\gamma+\epsilon_{A,t}}\widehat{Y}_t/\widehat{Y}_{t-1}\\
P_t/P_{t-1} &=& (\widehat{P}_t/\widehat{P}_{t-1})(m_{t-1}/e^{\gamma+\epsilon_{A,t}})
\end{eqnarray*}
where, importantly, hats over variables no longer mean deviations from steady state, but instead represent variables that have been made stationary. We come back to this important topic in details in section \ref{sec:nonstat} below. For now, we pause a moment to give some intuition for the above equations. In order, these equations correspond to:
\begin{enumerate}
\item The Euler equation in the goods market, representing the tradeoff to the economy of moving consumption goods across time.
\item The firms' borrowing constraint, also affecting labor demand, as firms use borrowed funds to pay for labor input.
\item The intertemporal labor market optimality condition, linking labor supply, labor demand, and the marginal rate of substitution between consumption and leisure. 
\item The equilibrium interest rate in which the marginal revenue product of labor equals the cost of borrowing to pay for that additional unit of labor.
\item The Euler equation in the credit market, which ensures that giving up one unit of consumption today for additional savings equals the net present value of future consumption.
\item The aggregate resource constraint.
\item The money market equilibrium condition equating nominal consumption demand to money demand to money supply to current nominal balances plus money injection. 
\item The credit market equilibrium condition.
\item The production function.
\item The stochastic process for money growth.
\item The stochastic process for technology.
\item The relationship between observable variables and stationary variables; more details on these last two equations appear in the following section. 
\end{enumerate}

\subsection{Declaring variables and parameters}
This block of the .mod file follows the usual conventions and would look like:\\
\\
\texttt{var m P c e W R k d n l Y\_obs P\_obs y dA; \\
varexo e\_a e\_m;\\
\\
parameters alp, bet, gam, mst, rho, psi, del;}\\
\\
where the choice of upper and lower case letters is not significant, the first set of endogenous variables, up to $l$, are as specified in the model setup above, and where the last five variables are defined and explained in more details in the section below on declaring the model in Dynare. The exogenous variables are as expected and concern the shocks to the evolution of technology and money balances. \\

\subsection{The origin of non-stationarity} \label{sec:nonstat}
The problem of non-stationarity comes from having stochastic trends in technology and money. The non-stationarity comes out clearly when attempting to solve the model for a steady state and realizing it does not have one. It can be shown that when shocks are null, real variables grow with $A_t$ (except for labor, $N_t$, which is stationary as there is no population growth), nominal variables grow with $M_t$ and prices with $M_t/A_t$. \textbf{Detrending} therefore involves the following operations (where hats over variables represent stationary variables): for real variables, $\hat q_t=q_t/A_t$, where $q_t = [y_t, c_t, i_t, k_{t+1} ]$. For nominal variables, $\hat{Q}_t = Q_t/M_t$, where $Q_t = [ d_t, l_t, W_t ]$. And for prices, $\hat P_t = P_t \cdot A_t / M_t$. \\

\subsection{Stationarizing variables}
Let's illustrate this transformation on output, and leave the transformations of the remaining equations as an exercise, if you wish (\citet{CogleyNason1994} includes more details on the transformations of each equation). We stationarize output by dividing its real variables (except for labor) by $A_t$. We define $\widehat{Y}_t$ to equal $Y_t / A_t$ and $\widehat{K}_t$ as $K_t / A_t$. \textsf{\textbf{NOTE!}} Recall from section \ref{sec:modspe} in chapter \ref{ch:solbase}), that in Dynare variables take the time subscript of the period in which they are decided (in the case of the capital stock, today's capital stock is a result of yesterday's decision). Thus, in the output equation, we should actually work with $\widehat{K}_{t-1}=K_{t-1} / A_{t-1}$. The resulting equation made up of stationary variables is
\begin{eqnarray*}
\frac{Y_t}{A_t} & = & \left( \frac{K_{t-1}}{A_{t-1}} \right)^\alpha A_t^{1-\alpha} N_t^{1-\alpha} A_t^{-1} A_{t-1}^\alpha \\
\widehat{Y}_t & = & \widehat{K}_{t-1}^\alpha N_t^{1-\alpha} \left( \frac{A_t}{A_{t-1}} \right)^{-\alpha} \\
& = & \widehat{K}_{t-1}^\alpha N_t^{1-\alpha} \exp(-\alpha (\gamma + \epsilon_{A,t}))
\end{eqnarray*}
where we go from the second to the third line by taking the exponential of both sides of the equation of motion of technology.\\

The above is the equation we retain for the .mod file of Dynare into which we enter:\\
\\
\texttt{y=k(-1)\textasciicircum alp*n\textasciicircum (1-alp)*exp(-alp*(gam+e\_a))}\\
\\

The other equations are entered into the .mod file after transforming them in exactly the same way as the one above. A final transformation to consider, that turns out to be useful since we often deal with the growth rate of technology, is to define \\
\\
\texttt{dA = exp(gam+e\_a)}\\
\\
by simply taking the exponential of both sides of the stochastic process of technology defined in the model setup above. \\

\subsection{Linking stationary variables to the data}
And finally, we must make a decision as to our \textbf{non-stationary observations}. We could simply stationarize them by \textbf{working with rates of growth} (which we know are constant). In the case of output, the observable variable would become $Y_t /Y_{t-1}$. We would then have to relate this observable, call it $gy\_{obs}$, to our (stationary) model's variables $\widehat Y_t$ by using the definition that $ \widehat{Y}_t \equiv Y_t/ A_t$. Thus, we add to the model block of the .mod file: \\
\\
\texttt{gy\_obs = dA*y/y(-1);}\\
\\
where, the $y$ of the .mod file are the stationary $\widehat Y_t$.\\

But, we could also \textbf{work with non-stationary data in levels}. This complicates things somewhat, but illustrates several features of Dynare worth highlighting; we therefore follow this path in the remainder of the example. The result is not very different, though, from what we just saw above. The goal is to add a line to the model block of our .mod file that relates the non stationary observables, call them $Y_{obs}$, to our stationary output, $\widehat Y_t$. We could simply write $Y_{obs} = \widehat Y_t A_t$. But since we don't have an $A_t$ variable, but just a $dA_t$, we we-write the above relationship in ratios. To the .mod file, we therefore add:\\
\\
\texttt{Y\_obs/Y\_obs(-1) = dA*y/y(-1);}\\
\\
We of course do the same for prices, our other observable variable, except that we use the relationship $P_{obs} = \widehat P_t M_t/A_t$ as noted earlier. The details of the correct transformations for prices are left as an exercise and can be checked against the results below.\\

\subsection{The resulting model block of the .mod file}

\texttt{model;\\
dA = exp(gam+e\_a);\\
log(m) = (1-rho)*log(mst) + rho*log(m(-1))+e\_m;\\
-P/(c(+1)*P(+1)*m)+bet*P(+1)*(alp*exp(-alp*(gam+log(e(+1))))*k\textasciicircum (alp-1)\\
*n(+1)\textasciicircum (1-alp)+(1-del)*exp(-(gam+log(e(+1)))))/(c(+2)*P(+2)*m(+1))=0;\\
W = l/n;\\
-(psi/(1-psi))*(c*P/(1-n))+l/n = 0;\\
R = P*(1-alp)*exp(-alp*(gam+e\_a))*k(-1)\textasciicircum alp*n\textasciicircum (-alp)/W;\\
1/(c*P)-bet*P*(1-alp)*exp(-alp*(gam+e\_a))*k(-1)\textasciicircum alp*n\textasciicircum (1-alp)/\\
(m*l*c(+1)*P(+1)) = 0;\\
c+k = exp(-alp*(gam+e\_a))*k(-1)\textasciicircum alp*n\textasciicircum (1-alp)+(1-del)\\
*exp(-(gam+e\_a))*k(-1);\\
P*c = m;\\
m-1+d = l;\\
e = exp(e\_a);\\
y = k(-1)\textasciicircum alp*n\textasciicircum (1-alp)*exp(-alp*(gam+e\_a));\\
Y\_obs/Y\_obs(-1) = dA*y/y(-1);\\
P\_obs/P\_obs(-1) = (p/p(-1))*m(-1)/dA;\\
end;}\\
\\
where, of course, the input conventions, such as ending lines with semicolons and indicating the timing of variables in parentheses, are the same as those listed in chapter \ref{ch:solbase}. \\

\textsf{\textbf{TIP!}} In the above model block, notice that what we have done is in fact relegated the non-stationarity of the model to just the last two equations, concerning the observables which are, after all, non-stationary. The problem that arises, though, is that we cannot linearize the above system in levels, as the last two equations don't have a steady state. If we first take logs, though, they become linear and it doesn't matter anymore where we calculate their derivative when taking a Taylor expansion of all the equations in the system. Thus, \textbf{when dealing with non-stationary observations, you must log-linearize your model} (and not just linearize it); this is a point to which we will return later. 

\subsection{Declaring observable variables}
We begin by declaring which of our model's variables are observables. In our .mod file we write\\
\\
\texttt{varobs P\_obs Y\_obs;}\\
\\
to specify that our observable variables are indeed $P\_obs$ and $Y\_obs$ as noted in the section above. \textsf{\textbf{NOTE!}} Recall from earlier that the number of observed variables must be smaller or equal to the number of shocks such that the model be estimated. If this is not the case, you should add measurement shocks to your model where you deem most appropriate. \\

\subsection{Declaring trends in observable variables}

Recall that we decided to work with the non-stationary observable variables in levels. Both output and prices exhibit stochastic trends. This can be seen explicitly by taking the difference of logs of output and prices to compute growth rates. In the case of output, we make use of the usual (by now!) relationship $Y_t=\widehat Y_t \cdot A_t$. Taking logs of both sides and subtracting the same equation scrolled back one period, we find:
\[
\Delta \ln Y_t = \Delta \ln \widehat Y_t + \gamma + \epsilon_{A,t}
\]
emphasizing clearly the drift term $\gamma$, whereas we know $\Delta \ln \widehat Y_t$ is stationary in steady state. \\

In the case of prices, we apply the same manipulations to show that:
\[
\Delta \ln P_t = \Delta \ln \widehat P_t + \ln m_{t-1} - \gamma - \epsilon_{A,t}
\]

Note from the original equation of motion of $\ln m_t$ that in steady state, $\ln m_t=\ln m^*$, so that the drift terms in the above equation are $\ln m^* - \gamma$.\footnote{This can also be see from substituting for $\ln m_{t-1}$ in the above equation with the equation of motion of $\ln m_t$ to yield: $\Delta \ln P_t = \Delta \ln \widehat P_t + \ln m^* + \rho(\ln m_{t-2}-\ln m^*) + \epsilon_{M,t} - \gamma - \epsilon_{A,t}$ where all terms on the right hand side are constant, except for $\ln m^*$ and $\gamma$.} \\

In Dynare, any trends, whether deterministic or stochastic (the drift term) must be declared up front. In the case of our example, we therefore write (in a somewhat cumbersome manner)\\
\\
\texttt{observation\_trends;\\
P\_obs (log(mst)-gam);\\
Y\_obs (gam);\\
end;}\\

In general, the command \texttt{observation\_trends} specifies linear trends as a function of model parameters for the observed variables in the model.\\

\subsection{Declaring unit roots in observable variables}

And finally, since P\_obs and Y\_obs inherit the unit root characteristics of their driving variables, technology and money, we must tell Dynare to use a diffuse prior (infinite variance) for their initialization in the Kalman filter. Note that for stationary variables, the unconditional covariance matrix of these variables is used for initialization. The algorithm to compute a true diffuse prior is taken from \citet{DurbinKoopman2001}. To give these instructions to Dynare, we write in the .mod\\
\\
\texttt{unit\_root\_vars P\_obs Y\_obs;}\\
\\
\textsf{\textbf{NOTE!}} You don't need to declare unit roots for any non-stationary model. Unit roots are only related to stochastic trends. You don't need to use a diffuse initial condition in the case of a deterministic trend, since the variance is finite.\\

\subsection{Specifying the steady state}
Declaring the steady state is just as explained in details and according to the same syntax explained in chapter \ref{ch:solbase}, covering the \texttt{initval}, \texttt{steady} and \texttt{check} commands. In chapter \ref{ch:estbase}, section \ref{sec:ssest}, we also discussed the usefulness of providing an external Matlab file to solve for the steady state. In this case, you can find the corresponding steady state file in the \textsl{models} folder under \textsl{UserGuide}. The file is called \textsl{fs2000ns\_steadystate.m}. There are some things to notice. First, the output of the function is the endogenous variables at steady state, the \texttt{ys} vector. The \texttt{check=0} limits steady state values to real numbers. Second, notice the declaration of parameters at the beginning; intuitive, but tedious... This functionality may be updated in later versions of Dynare. Third, note that the file is really only a sequential set of equalities, defining each variable in terms of parameters or variables solved in the lines above. So far, nothing has changed with respect to the equivalent file of chapter \ref{ch:estbase}. The only novelty is the declaration of the non-stationary variables, $P\_obs$ and $Y\_obs$ which take the value of 1. This is Dynare convention and must be the case for all your non-stationary variables. 

\subsection{Declaring priors}
We expand our .mod file with the following information: \\
\\
\texttt{estimated\_params;\\
alp, beta\_pdf, 0.356, 0.02; \\ 
bet, beta\_pdf, 0.993, 0.002; \\
gam, normal\_pdf, 0.0085, 0.003; \\
mst, normal\_pdf, 1.0002, 0.007; \\
rho, beta\_pdf, 0.129, 0.223;\\
psi, beta\_pdf, 0.65, 0.05;\\
del, beta\_pdf, 0.01, 0.005;\\
stderr e\_a, inv\_gamma\_pdf, 0.035449, inf;\\
stderr e\_m, inv\_gamma\_pdf, 0.008862, inf;\\
end;}\\
\\

\subsection{Launching the estimation}
We add the following commands to ask Dynare to run a basic estimation of our model:\\
\\
\texttt{estimation(datafile=fsdat,nobs=192,loglinear,mh\_replic=2000,\\
mode\_compute=4,mh\_nblocks=2,mh\_drop=0.45,mh\_jscale=0.65);}\\

\textsf{\textbf{NOTE!}} As mentioned earlier, we need to instruct Dynare to log-linearize our model, since it contains non-linear equations in non-stationary variables. A simple linearization would fail as these variables do not have a steady state. Fortunately, taking the log of the equations involving non-stationary variables does the job of linearizing them.\\

\subsection{The complete .mod file}
We have seen each part of the .mod separately; it's now time to get a picture of what the complete file looks like. For convenience, the file also appears in the \textsl{models} folder under \textsl{UserGuide} in your Dynare installation. The file is called \texttt{fs2000ns.mod}. \\
\\
\texttt{var m P c e W R k d n l Y\_obs P\_obs y dA; \\
varexo e\_a e\_m;\\
\\
parameters alp, bet, gam, mst, rho, psi, del;
\\
model;\\
dA = exp(gam+e\_a);\\
log(m) = (1-rho)*log(mst) + rho*log(m(-1))+e\_m;\\
-P/(c(+1)*P(+1)*m)+bet*P(+1)*(alp*exp(-alp*(gam+log(e(+1))))*k\textasciicircum (alp-1)\\
*n(+1)\textasciicircum (1-alp)+(1-del)*exp(-(gam+log(e(+1)))))/(c(+2)*P(+2)*m(+1))=0;\\
W = l/n;\\
-(psi/(1-psi))*(c*P/(1-n))+l/n = 0;\\
R = P*(1-alp)*exp(-alp*(gam+e\_a))*k(-1)\textasciicircum alp*n\textasciicircum (-alp)/W;\\
1/(c*P)-bet*P*(1-alp)*exp(-alp*(gam+e\_a))*k(-1)\textasciicircum alp*n\textasciicircum (1-alp)/(m*l*c(+1)*P(+1)) = 0;\\
c+k = exp(-alp*(gam+e\_a))*k(-1)\textasciicircum alp*n\textasciicircum (1-alp)+(1-del)*exp(-(gam+e\_a))*k(-1);\\
P*c = m;\\
m-1+d = l;\\
e = exp(e\_a);\\
y = k(-1)\textasciicircum alp*n\textasciicircum (1-alp)*exp(-alp*(gam+e\_a));\\
Y\_obs/Y\_obs(-1) = dA*y/y(-1);\\
P\_obs/P\_obs(-1) = (p/p(-1))*m(-1)/dA;\\
end;\\
\\
varobs P\_obs Y\_obs;\\
\\
observation\_trends;\\
P\_obs (log(mst)-gam);\\
Y\_obs (gam);\\
end;\\
\\
unit\_root\_vars = P\_obs Y\_obs;\\
\\
initval;\\
k = 6;\\
m = mst;\\
P = 2.25;\\
c = 0.45;\\
e = 1;\\
W = 4;\\
R = 1.02;\\
d = 0.85;\\
n = 0.19;\\
l = 0.86;\\
y = 0.6;\\
dA = exp(gam);\\
end;\\
\\
// the above is really only useful if you want to do a stoch\_simul\\
// of your model, since the estimation will use the Matlab\\
// steady state file also provided and discussed above.\\
\\
steady;\\
\\
estimated\_params;\\
alp, beta\_pdf, 0.356, 0.02; \\ 
bet, beta\_pdf, 0.993, 0.002; \\
gam, normal\_pdf, 0.0085, 0.003; \\
mst, normal\_pdf, 1.0002, 0.007; \\
rho, beta\_pdf, 0.129, 0.223;\\
psi, beta\_pdf, 0.65, 0.05;\\
del, beta\_pdf, 0.01, 0.005;\\
stderr e\_a, inv\_gamma\_pdf, 0.035449, inf;\\
stderr e\_m, inv\_gamma\_pdf, 0.008862, inf;\\
end;\\
\\
estimation(datafile=fsdat,nobs=192,loglinear,mh\_replic=2000,\\
mode\_compute=4,mh\_nblocks=2,mh\_drop=0.45,mh\_jscale=0.65);}\\
\\

\subsection{Summing it up}
The explanations given above of each step necessary to translate the \citet{Schorfheide2000} example into language that Dynare can understand and process was quite lengthy and involved a slew of new commands and information. It may therefore be useful, to gain a ``bird's eyeview'' on what we have just accomplished, and summarize the most important steps at a high level. This is done in figure \ref{fig:estsumm}.\\
\begin{figure} \label{fig:estsumm}
\begin{center} 
\includegraphics[width=1.0\textwidth]{P_flowest} 
\end{center} 
\caption[Steps of model estimation]{At a high level, there are five basic steps to translate a model into Dynare for successful estimation.} 
\end{figure}\\


\section{Comparing models based on their posterior distributions}
** TBD

\section{Where is your output stored?}
The output from estimation can be extremely varied, depending on the instructions you give Dynare. The \href{http://www.dynare.org/documentation-and-support/manual}{Reference Manual} overviews the complete set of potential output files and describes where you can find each one. 
\chapter{Solving DSGE models - Behind the scenes of Dynare} \label{ch:solbeh}

\section{Introduction}
The aim of this chapter is to peer behind the scenes of Dynare, or under its hood, to get an idea of the methodologies and algorithms used in its computations. Going into details would be beyond the scope of this User Guide which will instead remain at a high level. What you will find below will either comfort you in realizing that Dynare does what you expected of it - and what you would have also done if you had had to code it all yourself (with a little extra time on your hands!), or will spur your curiosity to have a look at more detailed material. If so, you may want to go through Michel Juillard's presentation on solving DSGE models to a first and second order (available on Michel Juillard's \href{http://jourdan.ens.fr/~michel/}{website}), or read \citet{CollardJuillard2001b} or \citet{SchmittGrohe2004} which gives a good overview of the most recent solution techniques based on perturbation methods. Finally, note that in this chapter we will focus on stochastic models - which is where the major complication lies, as explained in section \ref{sec:detstoch} of chapter \ref{ch:solbase}. For more details on the Newton-Raphson algorithm used in Dynare to solve deterministic models, see \citet{Juillard1996}. \\

\section{What is the advantage of a second order approximation?}
As noted in chapter \ref{ch:solbase} and as will become clear in the section below, linearizing a system of equations to the first order raises the issue of certainty equivalence. This is because only the first moments of the shocks enter the linearized equations, and when expectations are taken, they disappear. Thus, unconditional expectations of the endogenous variables are equal to their non-stochastic steady state values. \\

This may be an acceptable simplification to make. But depending on the context, it may instead be quite misleading. For instance, when using second order welfare functions to compare policies, you also need second order approximations of the policy function. Yet more clearly, in the case of asset pricing models, linearizing to the second order enables you to take risk (or the variance of shocks) into consideration - a highly desirable modeling feature. It is therefore very convenient that Dynare allows you to choose between a first or second order linearization of your model in the option of the \texttt{stoch\_simul} command. \\

\section{How does dynare solve stochastic DSGE models?}
In this section, we shall briefly overview the perturbation methods employed by Dynare to solve DSGE models to a first order approximation. The second order follows very much the same approach, although at a higher level of complexity. The summary below is taken mainly from Michel Juillard's presentation ``Computing first order approximations of DSGE models with Dynare'', which you should read if interested in particular details, especially regarding second order approximations (available on Michel Juillard's \href{http://jourdan.ens.fr/~michel/}{website}). \\

To summarize, a DSGE model is a collection of first order and equilibrium conditions that take the general form: 
\[
\mathbb{E}_t\left\{f(y_{t+1},y_t,y_{t-1},u_t)\right\}=0
\]
\begin{eqnarray*}
\mathbb{E}(u_t) &=& 0\\
\mathbb{E}(u_t u_t') &=& \Sigma_u
\end{eqnarray*}
and where:
\begin{description}
  \item[$y$]: vector of endogenous variables of any dimension
  \item[$u$]: vector of exogenous stochastic shocks of any dimension
\end{description}

The solution to this system is a set of equations relating variables in the current period to the past state of the system and current shocks, that satisfy the original system. This is what we call the policy function. Sticking to the above notation, we can write this function as:
\[
y_t = g(y_{t-1},u_t)
\]

Then, it is straightforward to re-write $y_{t+1}$ as
\begin{eqnarray*}
  y_{t+1} &=& g(y_t,u_{t+1})\\
  &=& g(g(y_{t-1},u_t),u_{t+1})\\
\end{eqnarray*}

We can then define a new function $F$, such that:
\[
F(y_{t-1},u_t,u_{t+1}) =
f(g(g(y_{t-1},u_t),u_{t+1}),g(y_{t-1},u_t),y_{t-1},u_t)\\
\]
which enables us to rewrite our system in terms of past variables, and current and future shocks:
\[
\mathbb{E}_t\left[F(y_{t-1},u_t,u_{t+1})\right] = 0
\]

We then venture to linearize this model around a steady state defined as:
\[
f(\bar y, \bar y, \bar y, 0) = 0
\]
having the property that:
\[
\bar y = g(\bar y, 0)
\]

The first order Taylor expansion around $\bar y$ yields:
\begin{eqnarray*}
\lefteqn{\mathbb{E}_t\left\{F^{(1)}(y_{t-1},u_t,u_{t+1})\right\} =}\\
&& \mathbb{E}_t\Big[f(\bar y, \bar y, \bar y)+f_{y_+}\left(g_y\left(g_y\hat y+g_uu \right)+g_u u' \right)\\
&& + f_{y_0}\left(g_y\hat y+g_uu \right)+f_{y_-}\hat y+f_u u\Big]\\
&& = 0
\end{eqnarray*}
with $\hat y = y_{t-1} - \bar y$, $u=u_t$, $u'=u_{t+1}$, $f_{y_+}=\frac{\partial f}{\partial y_{t+1}}$, $f_{y_0}=\frac{\partial f}{\partial y_t}$, $f_{y_-}=\frac{\partial f}{\partial y_{t-1}}$, $f_{u}=\frac{\partial f}{\partial u_t}$, $g_y=\frac{\partial g}{\partial y_{t-1}}$, $g_u=\frac{\partial g}{\partial u_t}$.\\

Taking expectations (we're almost there!):
\begin{eqnarray*}
   \lefteqn{\mathbb{E}_t\left\{F^{(1)}(y_{t-1},u_t, u_{t+1})\right\} =}\\
&& f(\bar y, \bar y, \bar y)+f_{y_+}\left(g_y\left(g_y\hat y+g_uu \right) \right)\\
&& + f_{y_0}\left(g_y\hat y+g_uu \right)+f_{y_-}\hat y+f_u u\Big\}\\
&=& \left(f_{y_+}g_yg_y+f_{y_0}g_y+f_{y_-}\right)\hat y+\left(f_{y_+}g_yg_u+f_{y_0}g_u+f_{u}\right)u\\
&=& 0\\
\end{eqnarray*}

As you can see, since future shocks only enter with their first moments (which are zero in expectations), they drop out when taking expectations of the linearized equations. This is technically why certainty equivalence holds in a system linearized to its first order. The second thing to note is that we have two unknown variables in the above equation: $g_y$ and $g_u$ each of which will help us recover the policy function $g$. \\

Since the above equation holds for any $\hat y$ and any $u$, each parenthesis must be null and we can solve each at a time. The first, yields a quadratic equation  in $g_y$, which we can solve with a series of algebraic trics that are not all immediately apparent (but detailed in Michel Juillard's presentation). Incidentally, one of the conditions that comes out of the solution of this equation is the Blanchard-Kahn condition: there must be as many roots larger than one in modulus as there are forward-looking variables in the model. Having recovered $g_y$, recovering $g_u$ is then straightforward from the second parenthesis. \\

Finally, notice that a first order linearization of the function $g$ yields:
\[
y_t = \bar y+g_y\hat y+g_u u
\]
And now that we have $g_y$ and $g_u$, we have solved for the (approximate) policy (or decision) function and have succeeded in solving our DSGE model. If we were interested in impulse response functions, for instance, we would simply iterate the policy function starting from an initial value given by the steady state. \\

The second order solution uses the same ``perturbation methods'' as above (the notion of starting from a function you can solve - like a steady state - and iterating forward), yet applies more complex algebraic techniques to recover the various partial derivatives of the policy function. But the general approach is perfectly isomorphic. Note that in the case of a second order approximation of a DSGE model, the variance of future shocks remains after taking expectations of the linearized equations and therefore affects the level of the resulting policy function.\\

\chapter{Estimating DSGE models - Behind the scenes of Dynare} \label{ch:estbeh}

This chapter focuses on the theory of Bayesian estimation. It begins by motivating Bayesian estimation by suggesting some arguments in favor of it as opposed to other forms of model estimation. It then attempts to shed some light on what goes on in Dynare's machinery when it estimates DSGE models. To do so, this section surveys the methodologies adopted for Bayesian estimation, including defining what are prior and posterior distributions, using the Kalman filter to find the likelihood function, estimating the posterior function thanks to the Metropolis-Hastings algorithm, and comparing models based on posterior distributions.


\section{Advantages of Bayesian estimation}

Bayesian estimation is becoming increasingly popular in the field of macro-economics. Recent papers have attracted significant attention; some of these include:  \citet{Schorfheide2000} which uses Bayesian methods to compare the fit of two competing DSGE models of consumption, \citet{LubikSchorfheide2003} which investigates whether central banks in small open economies respond to exchange rate movements, \citet{SmetsWouters2003} which applies Bayesian estimation techniques to a model of the Eurozone, \citet{Ireland2004} which emphasizes instead maximum likelihood estimation, \citet{VillaverdeRubioRamirez2004} which reviews the econometric properties of Bayesian estimators and compare estimation results with maximum likelihood and BVAR methodologies, \citet{LubikSchorfheide2005} which applies Bayesian estimation methods to an open macro model focussing on issues of misspecification and identification, and finally \citet{RabanalRubioRamirez2005} which compares the fit, based on posterior distributions, of four competing specifications of New Keynesian monetary models with nominal rigidities.\\

There are a multitude of advantages of using Bayesian methods to estimate a model, but five of these stand out as particularly important and general enough to mention here.\\

First, Bayesian estimation fits the complete, solved DSGE model, as opposed to GMM estimation which is based on particular equilibrium relationships such as the Euler equation in consumption. Likewise, estimation in the Bayesian case is based on the likelihood generated by the DSGE system, rather than the more indirect discrepancy between the implied DSGE and VAR impulse response functions. Of course, if your model is entirely mis-specified, estimating it using Bayesian techniques could be a disadvantage.\\

Second, Bayesian techniques allow the consideration of priors which work as weights in the estimation process so that the posterior distribution avoids peaking at strange points where the likelihood peaks. Indeed, due to the stylized and often misspecified nature of DSGE models, the likelihood often peaks in regions of the parameter space that are contradictory with common observations, leading to the ``dilemma of absurd parameter estimates''.\\

Third, the inclusion of priors also helps identifying parameters. Unfortunately, when estimating a model, the problem of identification often arises. It can be summarized by different values of structural parameters leading to the same joint distribution for observables. More technically, the problem arises when the posterior distribution is flat over a subspace of parameter values. But the weighting of the likelihood with prior densities often leads to adding just enough curvature in the posterior distribution to facilitate numerical maximization.\\

Fourth, Bayesian estimation explicitly addresses model misspecification by including shocks, which can be interpreted as observation errors, in the structural equations.\\

Sixth, Bayesian estimation naturally leads to the comparison of models based on fit. Indeed, the posterior distribution corresponding to competing models can easily be used to determine which model best fits the data. This procedure, as other topics mentioned above, is discussed more technically in the subsection below. 


\section{The basic mechanics of Bayesian estimation}
This and the following subsections are based in great part on work by, and discussions with, St�phane Adjemian, a member of the Dynare development team. Some of this work, although summarized in presentation format, is available in the ``conferences and workshops'' page of the \href{http://www.cepremap.cnrs.fr/juillard/mambo/index.php?option=com_content&task=blogsection&id=16&Itemid=94}{Dynare website}. Other helpful material includes \citet{AnSchorfheide2006}, which includes a clear and quite complete introduction to Bayesian estimation, illustrated by the application of a simple DSGE model. Also, the appendix of \citet{Schorfheide2000} contains details as to the exact methodology and possible difficulties encountered in Bayesian estimation. You may also want to take a glance at \citet{Hamilton1994}, chapter 12, which provides a very clear, although somewhat outdated, introduction to the basic mechanics of Bayesian estimation. Finally, the websites of \href{http://www.econ.upenn.edu/~schorf/}{Frank Schorfheide} and \href{http://www.econ.upenn.edu/~jesusfv/index.html}{Jesus Fernandez-Villaverde} contain a wide variety of very helpful material, from example files to lecture notes to related papers. Finally, remember to also check the \href{http://www.cepremap.cnrs.fr/juillard/mambo/index.php?option=com_forum&Itemid=95&page=viewforum&f=2&sid=164275ffd060698c8150318e8d6b453e}{open online examples} of the Dynare website for examples of .mod files touching on Bayesian estimation. \\

At its most basic level, Bayesian estimation is a bridge between calibration and maximum likelihood. The tradition of calibrating models is inherited through the specification of priors. And the maximum likelihood approach enters through the estimation process based on confronting the model with data. Together, priors can be seen as weights on the likelihood function in order to give more importance to certain areas of the parameter subspace. More technically, these two building blocks - priors and likelihood functions - are tied together by Bayes' rule. Let's see how. \\

First, priors are described by a density function of the form  
\[ p(\boldsymbol\theta_{\mathcal A}|\mathcal A) \] where $\mathcal A$ stands for a specific model, $\boldsymbol\theta_{\mathcal A}$ represents the parameters of model $\mathcal A$, $p(\bullet)$ stands for a probability density function (pdf) such as a normal, gamma, shifted gamma, inverse gamma, beta, generalized beta, or uniform function. \\

Second, the likelihood function describes the density of the observed
        data, given the model and its parameters:
        \[
            {\mathcal L}(\boldsymbol\theta_{\mathcal A}|\mathbf Y_T,\mathcal A) \equiv p(\mathbf Y_T | \boldsymbol\theta_{\mathcal A}, \mathcal A)
        \]
        where $\mathbf Y_T$ are the observations until period
        $T$, and where in our case the likelihood is recursive and can be written as:
        \[
            p(\mathbf Y_T | \boldsymbol\theta_{\mathcal A}, \mathcal A) =
            p(y_0|\boldsymbol\theta_{\mathcal A},\mathcal A)\prod^T_{t=1}p(y_t | \mathbf Y_{t-1},\boldsymbol\theta_{\mathcal A}, \mathcal A)
        \]

We now take a step back. Generally speaking, we have a prior density $p(\boldsymbol\theta)$ on the one hand, and on the other, a likelihood $p(\mathbf Y_T | \boldsymbol\theta)$. In the end, we are interested in $p(\boldsymbol\theta | \mathbf Y_T)$, the \textbf{posterior
        density}. Using the \textbf{Bayes theorem} twice we obtain this density of parameters
        knowing the data. Generally, we have
         \[p(\boldsymbol\theta | \mathbf Y_T) = \frac{p(\boldsymbol\theta ; \mathbf Y_T) }{p(\mathbf Y_T)}\]

We also know that
    \[
        p(\mathbf Y_T |\boldsymbol\theta ) = \frac{p(\boldsymbol\theta ; \mathbf Y_T)}{p(\boldsymbol\theta)}
        \Leftrightarrow p(\boldsymbol\theta ; \mathbf Y_T) = p(\mathbf Y_T |\boldsymbol\theta)\times
        p(\boldsymbol\theta)
    \]

By using these identities, we can combine the \textbf{prior density} and the \textbf{likelihood function} discussed above to get the posterior density:
        \[
            p(\boldsymbol\theta_{\mathcal A} | \mathbf Y_T, \mathcal A) = \frac{p(\mathbf Y_T |\boldsymbol\theta_{\mathcal A}, \mathcal A)p(\boldsymbol\theta_{\mathcal A}|\mathcal A)}{p(\mathbf Y_T|\mathcal A)}
        \]
where $p(\mathbf Y_T|\mathcal A)$ is the \textbf{marginal density} of the data
        conditional on the model:
        \[
            p(\mathbf Y_T|\mathcal A) = \int_{\Theta_{\mathcal A}} p(\boldsymbol\theta_{\mathcal A} ; \mathbf Y_T
            |\mathcal A) d\boldsymbol\theta_{\mathcal A}
        \]

Finally, the \textbf{posterior kernel} (or un-normalized posterior density, given that the marginal density above is a constant or equal for any parameter), corresponds to the numerator of the posterior density:
        \[
            p(\boldsymbol\theta_{\mathcal A} | \mathbf Y_T, \mathcal A) \propto p(\mathbf Y_T |\boldsymbol\theta_{\mathcal A},
            \mathcal A)p(\boldsymbol\theta_{\mathcal A}|\mathcal A) \equiv \mathcal K(\boldsymbol\theta_{\mathcal A} | \mathbf Y_T, \mathcal A)
        \]
This is the fundamental equation that will allow us to rebuild all posterior moments of interest. The trick will be to estimate the likelihood function with the help of the \textbf{Kalman filter} and then simulate the posterior kernel using a sampling-like or Monte Carlo method such as the \textbf{Metropolis-Hastings}. These topics are covered in more details below. Before moving on, though, the subsection below gives a simple example based on the above reasoning of what we mean when we say that Bayesian estimation is ``somewhere in between calibration and maximum likelihood estimation''. The example is drawn from \citet{Zellner1971}, although other similar examples can be found in \citet{Hamilton1994}, chapter 12.\\

\subsection{Bayesian estimation: somewhere between calibration and maximum likelihood estimation - an example}

Suppose a data generating process $y_t = \mu + \varepsilon_t$ for $t=1,...,T$, where $\varepsilon_t \sim \mathcal{N}(0,1)$ is gaussian white noise. Then, the likelihood is given by
            \[
                p(\mathbf{Y}_T|\mu) =
                (2\pi)^{-\frac{T}{2}}e^{-\frac{1}{2}\sum_{t=1}^T(y_t-\mu)^2}
            \]
We know from the above that $\widehat{\mu}_{ML,T} = \frac{1}{T}\sum_{t=1}^T y_t \equiv
                \overline{y}$ and that $\mathbb{V}[\widehat{\mu}_{ML,T}] = \frac{1}{T}$. \\
                
In addition, let our prior be a gaussian distribution with expectation
            $\mu_0$ and variance $\sigma_{\mu}^2$. Then, the posterior density is defined, up to a constant, by:
            \[
                p\left(\mu|\mathbf{Y}_T\right) \propto
                (2\pi\sigma_{\mu}^2)^{-\frac{1}{2}}e^{-\frac{1}{2}\frac{(\mu-\mu_0)^2}{\sigma_{\mu}^2}}\times(2\pi)^{-\frac{T}{2}}e^{-\frac{1}{2}\sum_{t=1}^T(y_t-\mu)^2}
            \]
Or equivalently, $p\left(\mu|\mathbf{Y}_T\right) \propto
                e^{-\frac{\left(\mu-\mathbb{E}[\mu]\right)^2}{\mathbb{V}[\mu]}}$, with
            \[
                \mathbb{V}[\mu] = \frac{1}{\left(\frac{1}{T}\right)^{-1} +
                \sigma_{\mu}^{-2}}
            \]
            and
            \[
                \mathbb{E}[\mu] =
                \frac{\left(\frac{1}{T}\right)^{-1}\widehat{\mu}_{ML,T} +
                \sigma_{\mu}^{-2}\mu_0}{\left(\frac{1}{T}\right)^{-1} +
                \sigma_{\mu}^{-2}}
            \]

From this, we can tell that the posterior mean is a convex combination of the prior mean and the ML estimate. In particular, if $\sigma_{\mu}^2 \rightarrow \infty$ (ie, we have no prior information, so we just estimate the model) then $\mathbb{E}[\mu] \rightarrow \widehat{\mu}_{ML,T}$, the maximum likelihood estimator. But if $\sigma_{\mu}^2 \rightarrow 0$ (ie, we're sure of ourselves and we calibrate the parameter of interest, thus leaving no room for estimation) then $\mathbb{E}[\mu] \rightarrow \mu_0$, the prior mean. Most of the time, we're somewhere in the middle of these two extremes. 


\section{DSGE models and Bayesian estimation}
\subsection{Rewriting the solution to the DSGE model}
Recall from chapter \ref{ch:solbeh} that any DSGE model, which is really a collection of first order and equilibrium conditions, can be written in the form $\mathbb{E}_t\left\{f(y_{t+1},y_t,y_{t-1},u_t)\right\}=0$, taking as a solution equations of the type $y_t = g(y_{t-1},u_t)$, which we call the decision rule. In more appropriate terms for what follows, we can rewrite the solution to a DSGE model as a system in the following manner:
  \begin{eqnarray*}
    y^*_t &=& M\bar y(\theta)+M\hat y_t+N(\theta)x_t+\eta_t\\
\hat y_t &=& g_y(\theta)\hat y_{t-1}+g_u(\theta)u_t\\
E(\eta_t \eta_t') &=& V(\theta)\\
E(u_t u_t') &=& Q(\theta)
  \end{eqnarray*}
where $\hat y_t$ are variables in deviations from steady state, $\bar y$ is the vector of steady state values and $\theta$ the vector of deep (or structural) parameters to be estimated. Other variables are described below.\\

The second equation is the familiar decision rule mentioned above. But the equation expresses a relationship among true endogenous variables that are not directly observed. Only $y^*_t$ is observable, and it is related to the true variables with an error $\eta_t$. Furthermore, it may have a trend, which is captured with $N(\theta)x_t$ to allow for the most general case in which the trend depends on the deep parameters. The first and second equations above therefore naturally make up a system of measurement and transition or state equations, respectively, as is typical for a Kalman filter (you guessed it, it's not a coincidence!). \\

\subsection{Estimating the likelihood function of the DSGE model}
The next logical step is to estimate the likelihood of the DSGE solution system mentioned above. The first apparent problem, though, is that the equations are non linear in the deep parameters. Yet, they are linear in the endogenous and exogenous variables so that the likelihood may be evaluated with a linear prediction error algorithm like the Kalman filter. This is exactly what Dynare does. As a reminder, here's what the Kalman filter recursion does.\\

For $t=1,\ldots,T$ and with initial values $y_1$ and $P_1$ given, the recursion follows
\begin{eqnarray*}
  v_t &=& y^*_t - \bar y^* - M\hat y_t - Nx_t\\
  F_t &=& M P_t M'+V\\
  K_t &=& g_yP_tg_y'F_t^{-1}\\
  \hat y_{t+1} &=& g_y \hat y_t+K_tv_t\\
  P_{t+1} &=& g_y P_t (g_y-K_tM)'+g_uQg_u'
\end{eqnarray*}
For more details on the Kalman filter, see \citet{Hamilton1994}, chapter 13. \\

From the Kalman filter recursion, it is possible to derive the \textbf{log-likelihood} given by
\[
\ln \mathcal{L}\left(\boldsymbol\theta|\mathbf Y^*_T\right) = -\frac{Tk}{2}\ln(2\pi)-\frac{1}{2}\sum_{t=1}^T|F_t|-\frac{1}{2}v_t'F_t^{-1}v_t  
\]
where the vector $\boldsymbol\theta$ contains the parameters we have to estimate: $\theta$, $V(\theta)$ and $Q(\theta)$ and where $Y^*_T$ expresses the set of observable endogenous variables $y^*_t$ found in the measurement equation. \\

The log-likelihood above gets us one step closer to our goal of finding the posterior distribution of our parameters. Indeed, the \textbf{log posterior kernel} can be expressed as 
\[
\ln \mathcal{K}(\boldsymbol\theta|\mathbf Y^*_T) = \ln \mathcal{L}\left(\boldsymbol\theta|\mathbf Y^*_T\right) + \ln p(\boldsymbol\theta)
\]
where the first term on the right hand side is now known after carrying out the Kalman filter recursion. The second, recall, are the priors, and are also known. \\

\subsection{Finding the mode of the posterior distribution}
Next, to find the mode of the posterior distribution - a key parameter and an important output of Dynare - we simply maximize the above log posterior kernel with respect to $\theta$. This is done in Dynare using numerical methods. Recall that the likelihood function is not Gaussian with respect to $\theta$ but to functions of $\theta$ as they appear in the state equation. Thus, this maximization problem is not completely straightforward, but fortunately doable with modern computers. \\

\subsection{Estimating the posterior distribution}
Finally, we are now in a position to find the posterior distribution of our parameters. The distribution will be given by the kernel equation above, but again, it is a nonlinear and complicated function of the deep parameters $\theta$. Thus, we cannot obtain an explicit form for it. We resort, instead, to sampling-like methods, of which the Metropolis-Hastings has been retained in the literature as particularly efficient. This is indeed the method adopted by Dynare.\\

The general idea of the Metropolis-Hastings algorithm is to simulate the posterior distribution. It is a ``rejection sampling algorithm'' used to generate a sequence of samples (also known as a ``Markov Chain'' for reasons that will become apparent later) from a distribution that is unknown at the outset. Remember that all we have is the posterior mode; we are instead more often interested in the mean and variance of the estimators of $\theta$. To do so, the algorithm builds on the fact that under general conditions the distribution of the deep parameters will be asymptotically normal. The algorithm, in the words of An and Shorfheide, ``constructs a Gaussian approximation around the posterior mode and uses a scaled version of the asymptotic covariance matrix as the covariance matrix for the proposal distribution. This allows for an efficient exploration of the posterior distribution at least in the neighborhood of the mode'' (\citet{AnSchorfheide2006}, p. 18). More precisely, the \textbf{Metropolis-Hastings algorithm implements the following steps}: 
\begin{enumerate}
        \item Choose a starting point $\boldsymbol\theta^\circ$, where this is typically the posterior mode, and run a loop over
        2-3-4. 
        \item Draw a \emph{proposal} $\boldsymbol\theta^*$ from a \emph{jumping} distribution
        \[
            J(\boldsymbol\theta^*|\boldsymbol\theta^{t-1}) =
            \mathcal N(\boldsymbol\theta^{t-1},c\Sigma_{m})
        \]
        where $\Sigma_{m}$ is the inverse of the Hessian computed at the posterior mode.
        \item Compute the acceptance ratio
        \[
            r = \frac{p(\boldsymbol\theta^*|\mathbf Y_T)}{p(\boldsymbol\theta^{t-1}|\mathbf
            Y_T)} = \frac{\mathcal K(\boldsymbol\theta^*|\mathbf Y_T)}{\mathcal K(\boldsymbol\theta^{t-1}|\mathbf
            Y_T)}
        \]
        \item Finally accept or discard the proposal $\boldsymbol\theta^*$  according to the following rule, and update, if necessary, the jumping distribution:
        \[
            \boldsymbol\theta^t = \left\{
            \begin{array}{ll}
                \boldsymbol\theta^* & \mbox{ with probability $\min(r,1)$}\\
                \boldsymbol\theta^{t-1} & \mbox{ otherwise.}
            \end{array}\right.
        \]
    \end{enumerate}

Figure \ref{fig:MH} tries to clarify the above. In step 1, choose a candidate paramter, $\theta^*$ from a Normal distribution, whose mean has been set to $\theta^{t-1}$ (this will become clear in just a moment). In step 2, compute the value of the posterior kernel for that candidate parameter, and compare it to the value of the kernel from the mean of the drawing distribution. In step 3, decide whether or not to hold on to your candidate parameter. If the acceptance ratio is greater than one, then definitely keep your candidate. Otherwise, go back to the candidate of last period (this is true in very coarse terms, notice that in fact you would keep your candidate only with a probability less than one). Then, do two things. Update the mean of your drawing distribution, and note the value of the parameter your retain. After having repeated these steps often enough, in the final step, build a histogram of those retained values. Of course, the point is for each ``bucket'' of the histogram to shrink to zero. This ``smoothed histogram'' will eventually be the posterior distribution after sufficient iterations of the above steps.\\
\begin{figure} \label{fig:MH}
\begin{center} 
\includegraphics[width=1.0\textwidth]{P_MH2} 
\end{center} 
\caption[Illustration of the Metropolis-Hastings algorithm]{The above sketches the Metropolis-Hastings algorithm, used to build the posterior distribution function. Imagine repeating these steps a large number of times, and smoothing the ``histogram'' such that each ``bucket'' has infinitely small width.} 
\end{figure}

But why have such a complicated acceptance rule? The point is to be able to visit the entire domain of the posterior distribution. We should not be too quick to simply throw out the candidate giving a lower value of the posterior kernel, just in case using that candidate for the mean of the drawing distribution allows us to to leave a local maximum and travel towards the global maximum. Metaphorically, the idea is to allow the search to turn away from taking a small step up, and instead take a few small steps down in the hope of being able to take a big step up in the near future. Of course, an important parameter in this searching procedure is the variance of the jumping distribution and in particular the \textbf{scale factor}. If the scale factor is too small, the \textbf{acceptance rate} (the fraction of candidate parameters that are accepted in a window of time) will be too high and the Markov Chain of candidate parameters will ``mix slowly'', meaning that the distribution will take a long time to converge to the posterior distribution since the chain is likely to get ``stuck'' around a local maximum. On the other hand, if the scale factor is too large, the acceptance rate will be very low (as the candidates are likely to land in regions of low probability density) and the chain will spend too much time in the tails of the posterior distribution. \\

While these steps are mathematically clear, at least to a machine needing to undertake the above calculations, several practical questions arise when carrying out Bayesian estimation. These include: How should we choose the scale factor $c$ (variance of the jumping
        distribution)? What is a satisfactory acceptance rate? How many draws are ideal? How is convergence of the Metropolis-Hastings iterations assessed? These are all important questions that will come up in your usage of Dynare. They are addressed as clearly as possible in section \ref{sec:estimate} of Chapter \ref{ch:estbase}. \\

\section{Comparing models using posterior distributions}
As mentioned earlier, while touting the advantages of Bayesian estimation, the posterior distribution offers a particularly natural method of comparing models. Let's look at an illustration. \\

Suppose we have a prior distribution over two competing models: $p(\mathcal{A})$ and $p(\mathcal{B})$. Using Bayes' rule, we can compute the posterior
            distribution over models, where $\mathcal{I}=\mathcal{A},\mathcal{B}$ 
        \[
            p(\mathcal{I}|\mathbf Y_T) = \frac{p(\mathcal{I})p(\mathbf Y_T|\mathcal{I})}
            {\sum_{\mathcal{I}=\mathcal{A},\mathcal{B}}p(\mathcal{I})p(\mathbf Y_T|\mathcal{I})}
        \]
where this formula may easily be generalized to a collection of $N$ models.

Then, the comparison of the two models is done very naturally through the ratio of the posterior model distributions. We call this the \textbf{posterior odds ratio}:
        \[
            \frac{p(\mathcal{A}|\mathbf Y_T)}{p(\mathcal{B}|\mathbf
            Y_T)} = \frac{p(\mathcal{A})}{p(\mathcal{B})}
            \frac{p(\mathbf Y_T|\mathcal{A})}{p(\mathbf Y_T|\mathcal{B})}
        \]

The only complication is finding the magrinal density of the data conditional on the model, $p(\mathbf Y_T|\mathcal{I})$, which is also the denominator of the posterior density $p(\boldsymbol\theta | \mathbf Y_T)$ discussed earlier. This requires some detailed explanations of their own. \\

For each model $\mathcal{I}=\mathcal{A},\mathcal{B}$ we can evaluate, at least theoretically, the marginal
        density of the data conditional on the model by integrating out the deep parameters $\boldsymbol\theta_{\mathcal{I}}$ from the posterior kernel:
        \[
            p(\mathbf Y_T|\mathcal{I}) = \int_{\Theta_{\mathcal{I}}} p(\boldsymbol\theta_{\mathcal{I}}; \mathbf Y_T
            |\boldsymbol\theta_{\mathcal{I}},\mathcal{I}) d\boldsymbol\theta_{\mathcal{I}} = \int_{\Theta_{\mathcal{I}}} p(\boldsymbol\theta_{\mathcal{I}}|\mathcal{I})\times p(\mathbf Y_T
            |\boldsymbol\theta_{\mathcal{I}},\mathcal{I}) d\boldsymbol\theta_{\mathcal{I}}
        \]
Note that the expression inside the integral sign is exactly the posterior kernel. To remind you of this, you may want to glance back at the first subsection above, specifying the basic mechanics of Bayesian estimation.\\
        
        To obtain the marginal density of the data conditional on the model, there are two options. The first is to assume a functional form of the posterior kernel that we can integrate. The most straightforward and the best approximation, especially for large samples, is the Gaussian (called a \textbf{Laplace approximation}). In this case, we would have the following estimator:
\[
\widehat{p}(\mathbf Y_T|\mathcal I) = (2\pi)^{\frac{k}{2}}|\Sigma_{\boldsymbol\theta^m_{\mathcal I}}|^{\frac{1}{2}}p(\boldsymbol\theta_{\mathcal I}^m|\mathbf
Y_T,\mathcal I)p(\boldsymbol\theta_{\mathcal I}^m|\mathcal I)
\]
where $\boldsymbol\theta_{\mathcal I}^m$ is the posterior mode. The advantage of this technique is its computational efficiency: time consuming Metropolis-Hastings iterations are not necessary, only the numerically calculated posterior mode is required. \\

The second option is instead to use information from the Metropolis-Hastings runs and is typically referred to as the \textbf{Harmonic Mean Estimator}. The idea is to simulate the marginal density of interest and to simply take an average of these simulated values. To start, note that
\[
p(\mathbf{Y}_T|\mathcal I)=\mathbb{E}\left[\frac{f(\boldsymbol\theta_{\mathcal I})}{p(\boldsymbol\theta_{\mathcal I}|\mathcal I)
            p(\mathbf{Y}_T|\boldsymbol\theta_{\mathcal I},\mathcal I)}\biggl|\boldsymbol\theta_{\mathcal I},\mathcal I\biggr.\right]^{-1}
\]
where $f$ is a probability density function, since 
\[
                \mathbb{E}\left[\frac{f(\boldsymbol\theta_{\mathcal I})}{p(\boldsymbol\theta_{\mathcal I}|\mathcal I)
                p(\mathbf{Y}_T|\boldsymbol\theta_{\mathcal I},\mathcal I)}\biggl|\boldsymbol\theta_{\mathcal I},\mathcal I\biggr.\right]
                =
                \frac{\int_{\Theta_{\mathcal I}}f(\boldsymbol\theta)d\boldsymbol\theta}{\int_{\Theta_{\mathcal I}}p(\boldsymbol\theta_{\mathcal I}|I)
                p(\mathbf{Y}_T|\boldsymbol\theta_{\mathcal I},\mathcal I)d\boldsymbol\theta_{\mathcal I}}
\]
and the numerator integrates out to one (see\citet{Geweke1999} for more details). \\

This suggests the following estimator of the marginal
        density
        \[
            \widehat{p}(\mathbf{Y}_T|\mathcal I)= \left[\frac{1}{B}\sum_{b=1}^B
            \frac{f(\boldsymbol\theta_{\mathcal I}^{(b)})}{p(\boldsymbol\theta_{\mathcal I}^{(b)}|\mathcal I)
            p(\mathbf{Y}_T|\boldsymbol\theta_{\mathcal I}^{(b)},\mathcal I)}\right]^{-1}
        \]
where each drawn vector $\boldsymbol\theta_{\mathcal I}^{(b)}$ comes from the
        Metropolis-Hastings iterations and where the probability density function $f$ can be viewed as a weights on the posterior kernel in order to downplay the importance of extreme values of $\boldsymbol\theta$. \citet{Geweke1999} suggests to use a truncated Gaussian function, leading to what is typically referred to as the \textbf{Modified Harmonic Mean Estimator}. \\    
            
            

\chapter{Optimal policy under commitment} \label{ch:ramsey}
\chapter{Troubleshooting} \label{ch:trouble}

To make sure this section is as user friendly as possible, the best is to compile what users have to say! Please let me know what your most common problem is with Dynare, how Dynare tells you about it and how you solve it. Thanks for your precious help!

\backmatter

\bibliography{DynareBib}
\bibliographystyle{econometrica}
%\printindex

\end{document}

